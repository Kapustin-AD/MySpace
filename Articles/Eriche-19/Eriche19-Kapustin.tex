\documentclass[serif,professionalfonts]{beamer}
%\usepackage[T2A]{fontenc}
\usepackage[utf8]{inputenc}
\usepackage[english]{babel}
\usepackage{amsmath}
\usepackage{amssymb}
\usepackage{amsfonts}
\usepackage{graphicx}
\usepackage{xfrac}
\usepackage{tikz}
\usepackage{libertine}

\usetheme{Madrid}
\setbeamertemplate{navigation symbols}{}
%\logo{\includegraphics[width=1cm]{eagle2.eps}}
\begin{document}
\title[Embedding theory and mimetic gravity]{Similarities between the embedding theory and the mimetic gravity}
\author[\textbf{Kapustin}, Paston]{\textbf{A.D. Kapustin}, S.A. Paston}
\titlegraphic{\includegraphics[width=2cm]{eagle2.eps}}
\institute[SPbU]{Saint Petersburg State University, \\
Department of High Energy and Elementary Particles physics.}
\date{}
\maketitle

\begin{frame}
\frametitle{Outline of the presentation}

\begin{itemize}
	\item{Mimetic gravity, embedding theory and their field equation}
	\item{Equivalent forms of actions}
	\item{Embedding gravity and mimetic limit}
	\item{Solution in case of weak gravitational field} 
\end{itemize}
\end{frame}

\begin{frame}
\frametitle{Mimetic gravity and Dark matter}

Action
\begin{equation}
	S^{EH} + S_m= \int d^4x \sqrt{-g}\left( \frac{R}{2 \varkappa} + \mathcal{L}_m \right)
\end{equation}
with substitution
\begin{equation}
	g_{\mu \nu} = \bar{g}_{\mu \nu} \bar{g}^{\alpha \beta} \partial_{\alpha} \lambda \partial_{\beta} \lambda
\end{equation}
leads to field equations
\begin{align}
	G_{\mu \nu} &= \varkappa \left(T_{\mu \nu} + n u_{\mu} u_{\nu} \right) \\
	D_{\mu} (n u^{\mu}) &= 0\\
	g^{\mu \nu} u_{\mu} u_{\nu} &= 1
\end{align}
with $n = \left( \frac{1}{\varkappa} G - T \right)$ and $u_{\mu} = \partial_{\mu} \lambda$

\

So the term $n u^{\mu} u^{\nu}$ corresponds to some dustlike ``dark'' matter ``without additional matter'' in theory

\textit{\small{A.H. Chamseddine, V. Mukhanov, JHEP, 2013:11 (2013), 135, arXiv:1308.5410}}
\end{frame}

\begin{frame}
\frametitle{Embedding theory}
\begin{equation}
	S^{EH} + S_m= \int d^4x \sqrt{-g}\left( \frac{R}{2 \varkappa} + \mathcal{L}_m \right)
\end{equation}
Einstein-Hilbert action with substitution
\begin{equation}
	g_{\mu \nu} = \partial_{\mu} y^a \partial_{\nu} y_a, \ \ \ \ a = 0, 1, \ldots , 9
\end{equation}
leads to Regge-Teitelboim equations
\begin{equation}
	D_{\mu} \left( (G^{\mu \nu}-\varkappa T^{\mu \nu})\partial_{\nu}y^a\right) = 0
\end{equation}
which could be rewritten in the form
\begin{align}
	G_{\mu \nu} &= \varkappa \left( T_{\mu \nu} + \tau_{\mu \nu} \right) \\
	D_{\mu} \left(\tau^{\mu \nu} \partial_{\nu} y^a \right)& = 0
\end{align}
\end{frame}

\begin{frame}
\frametitle{Similarity between the embedding and mimetic gravity}

The equivalent set of field equation could be obtained by using the additional action
\begin{equation}
	S^{EH} + S_m + S^{+}
\end{equation}
Among the different variants $S^{+}$ can be chosen in the similar form for both mimetic gravity and embedding theory

Mimetic gravity
\begin{equation}
	S^{+} = \int d^4x \sqrt{-g} \left( j^{\mu} \partial_{\mu}y - \sqrt{j^{\mu} g_{\mu \nu} j^{\nu}}\right)
\end{equation}
\textit{\small{S.A.Paston, Phys.Rev.D, 96 (2017), 084059, arXiv:1708.03944.}}

Embedding gravity
\begin{equation}
	S^{+} = \int d^4x \sqrt{-g} \left( j^{\mu}_{a} \partial_{\mu}y^a - tr \sqrt{\eta^{ab}j^{\mu}_{a} g_{\mu \nu} j^{\nu}_{c}}\right)
\end{equation}
\textit{\small{S.A.Paston, Eur. Phys. J. C, 78:12 (2018), 989, arXiv:1806.10902.}}
\end{frame}


\begin{frame}	
In the last form embedding gravity leads to the set of equations
\begin{align}
\label{y}	\partial_{\mu} y^a &= \beta^{-1}_{\mu \nu} j^{\nu a} \\
	D_{\mu} j^{\mu}_{a} &= 0 \\
	G_{\mu \nu} &= \varkappa \left( T_{\mu \nu} + \beta_{\mu \nu} \right) 
\end{align}
with $\beta_{\nu \alpha} = \sqrt{\eta^{ab}j^{\mu}_{a} g_{\mu \nu} j_{\alpha b}}$

\

If matrix $\beta$ has the rank $1$ the theory is equivalent to mimetic gravity with dust-like
EMT of additional matter

The most general limit giving the mimetic matter is
\begin{equation}
	j^{\mu}_a = j^{\mu} N_a + h^{\mu}_a, \ \ \ \ \text{with} \ \ h^{\mu}_a \to 0 
\end{equation}
with arbitrary vector $N_a$ in ambient space

There is a question what happens to geometry in this limit because equations (\ref{y}) defining embedding functions become singular
\end{frame}

\begin{frame}
\frametitle{Tetrad formalism and SVD}
It is convenient to study this limit using the tetrad formalism for GR. Then the metric is expressed through the tetrad $e_{\mu}^A$
\begin{equation}
	g_{\mu \nu} = e_{\mu}^A \eta_{AB} e_{\nu}^B
\end{equation}
where capital Latin characters correspond to gauge $SO(1,3)$ symmetry

The aim is to rewrite $j_{\mu}^a$ in the form
\begin{equation}
\label{j}
	j_{\mu}^a = \sum \limits_{A=0}^{3} \rho_{(A)} e_{\mu}^A n_A^a \ \ \ \ \text{with} \ \ n_A^a n_{Ba} = \eta_{AB}
\end{equation}
In Euclidean space it is always possible for every rectangular matrix due to Singular Value Decomposition,while in pseudo Euclidean space it requires additional restriction  

The mimetic limit then is reformulated in the form of $\rho_{(1,2,3)} \to 0$
\end{frame}

\begin{frame}
\frametitle{Solution in case of weak gravitational field}
Using the decomposition (\ref{j}) the field equations of the embedding gravity could be rewritten in the form
\begin{equation}
\label{20}
	\partial_{\mu} y^a = e_{\mu}^{\ A} n_{A}^{\ a}
\end{equation} 
\begin{equation}
\label{21}
	D_{\mu}\left( e^{\mu}_{\ A} \Lambda^{AB} n_{B}^{\ a} \right) = 0
\end{equation}
\begin{equation}
\label{22}
	G_{\mu \nu} = \varkappa \left( T_{\mu \nu} + e_{\mu}^{\ A} \Lambda_{A B} e_{\nu}^{\ B} \right)
\end{equation}
with $\Lambda^{A B} = diag(\rho_0, \rho_1, \rho_2, \rho_3)$ and covariant derivative $D_{\mu}$ defined by the following rules
\begin{equation}
	D_{\mu} \phi = \partial_{\mu} \phi, \ \ \ \ D_{\mu} k^{\nu} = \partial_{\mu} k^{\nu} + \Gamma^{\nu}_{\mu \sigma} k^{\sigma}, \ \ \ \ D_{\mu} k_{A} = \partial_{\mu} k_{A} + W_{\mu \ A}^{\ C} k_{C} 
\end{equation}
Here $W_{\mu \ A}^{\ C}$ is gauge connection and $\Gamma^{\nu}_{\mu \sigma}$ is Riemannian connection 
\end{frame}

\begin{frame}
We will solve this equations in the limit of small gravitational field, the mimetic limit $\rho_{1,2,3}=0$ and $T_{\mu \nu} = 0$ simultaniously

Then $g^{\mu \nu} = \eta^{\mu \nu} + \varkappa \, h^{\mu \nu}$, in static case $h^{00} = - 2 \varphi $
\begin{equation}
	e^{\mu}_{\ A}  = \delta_{A}^{\mu} + \frac{\varkappa}{2} \, h_{A}^{\mu} + \varkappa \, \alpha^{\mu} _{\ A}, \ \ \text{$h$ --- symmetric, $\alpha$ --- anti-symmetric}
\end{equation}
\begin{equation}
	n_{A}^{\ a} = \delta_{A}^{a} + \sqrt{\varkappa} \, \beta_{A}^{\ a} + \frac{\varkappa}{2} \, \beta_{A}^{\ b} \beta_{b}^{\ a}, \ \ \text{$\beta$ --- anti-symmetric}
\end{equation}
The gauge connections could be found from the restrictions $D_{\mu} e^{\nu}_{\ A} = 0$
\begin{equation}
	W_{\mu \ A}^{\ B} = \varkappa \, \left[ \frac{1}{2} \left( \partial_B h^{A}_{\mu} - \partial^{A} h_{B \mu} \right) - \partial_{\mu} \alpha^{A}_{\ B} \right]
\end{equation}
The equations (\ref{21}) --- (\ref{22}) in a first nontrivial order turns into a set of restrictions
\begin{equation}
\begin{split}
	\partial_0 \rho_0 = 0, \ \ \partial^{I} \left( \rho \, \alpha^{0}_{\ I} \right) &= 0, \ \ \partial_0 \alpha^{0}_{\ I} = -\partial_{I} \varphi \\
	\partial_0 \beta_{0}^{\ a} &= 0
\end{split}
\end{equation}
\end{frame}

\begin{frame}
	A remaining freedom in parameters $\alpha$ and $\beta$ could be found by solving the equation
\begin{equation}
	\partial_{\mu} y^a = e_{\mu}^{\ A} n_{A}^{\ a}
\end{equation}
which could be equivalently replaced by
\begin{equation}
	D_{\nu} e_{\mu}^{\ A} n_{A}^{\ a} - D_{\mu} e_{\nu}^{\ A} n_{A}^{\ a} = 0
\end{equation}

The particular solution with $\alpha^{I}_{\ K} = 0, \beta_{A}^{\ B} = 0$ leads to the embedding functions
\begin{align}
	y^0 &= t + t \, \varkappa \, \varphi + \frac{3}{2} \, \varkappa \, t \nonumber \\
	y^{I} &= x^{I} - \varkappa \, \frac{t}{2} \, \alpha^{I}_{\ 0} \nonumber \\
	y^4 = \sin{\omega_1(x)} \ \ \ \ y^6 &= \sin{\omega_2(x)} \ \ \ \ y^8 = \sin{\omega_3(x)} \\
	y^5 = \sin{\omega_1(x)} \ \ \ \ y^7 &= \sin{\omega_2(x)} \ \ \ \ y^9 = \sin{\omega_3(x)} \nonumber \\
	\text{where} \ \  \partial_I \partial_K \varphi &= \sum \limits_{n = 1}^{3} \partial_I \, \omega_n(x) \partial_K \, \omega_n(x) 
\end{align}

A general solution is expected to be found in the near future.
\end{frame}

\begin{frame}
\begin{center}
	\Large{Thank you for your attention!}
\end{center}
\end{frame}
\end{document}