\documentclass[12pt]{article}
\usepackage[utf8]{inputenc}
\usepackage[T2A]{fontenc}
%\usepackage{libertine}
%\usepackage{pscyr}
\usepackage[a4paper, left = 2.5cm,right = 2.5cm,top = 2.5cm,bottom = 3cm]{geometry}
\usepackage[english]{babel}
\usepackage{amsmath}
\usepackage{amssymb}
\usepackage{amsfonts}

\begin{document}
\title{Similarities between the embedding theory and mimetic gravity}
\date{}
\maketitle
\begin{abstract}
	Based on recent works on mimetic gravity and the embedding theory I want to talk about the similarity between this two approaches. In mimetic gravity, which has recently become popular, a special substitution for a metric leads to the appearance in theory of additional dust matter without changing the usual Einstein-Hilbert action. A similar substitution occurs when considering gravity in the form of an embedding theory. In this case, the space-time is considered to be embedded into a flat ambient space of a larger number of dimensions, and the metric induced. Such an approach may be interesting, since it has a clear geometric sense, and the arising matter has richer properties.
\end{abstract}
\end{document}