\documentclass[a4paper, 11pt]{article}
\usepackage{libertine}
\usepackage[english, russian]{babel}
\pagestyle{empty}
\begin{document}

	Уважаемая Кузнецова Галина Ивановна,
	
	\
	
	Благодарю Вас за активную позицию в этом вопросе! Что касается выставленных баллов, то действительно, работы с высокими баллами перепроверяются городским жюри, чтобы добиться равных условий для участников, потенциально проходящих на следующие этапы. При этом иногда возникают довольно существенные отличия от баллов районных методистов, причем как в большую, так и в меньшую сторону.
	
	Работу Дидиной Дарьи я, по Вашей просьбе, перепроверил. Выставленные баллы полностью соответствуют критериям, по которым проводилась перепроверка, и я могу коротко пояснить, за что были сняты баллы:
	\begin{itemize}
		\item{2 балла в первой задаче были сняты за неточную формулировку вывода в одном из случаев, что является хотя и строгим, но одинаково справедливым для всех условием.}
		\item{4 балла во второй задаче были сняты, так как в ходе всего решения использовались сильно округленные (до одной значащей цифры) вычисления, которые не являлись достаточно хорошим описанием процесса и привели к неправильному результату.}
		\item{7 баллов за четвертую задачу не были поставлены, так как все приведенное решение состояло из набора вычислений без каких-либо пояснений. Часть вычислений была зачеркнута полностью или наполовину, а решение в целом не позволяло определить правильность соображений, на которых оно было построено.}
		\item{Баллы за третью и пятую задачи изменены не были.}
	\end{itemize}
	
	Что касается неподабающего общения на апелляции, то я прошу прощения за возникновение такого прецедента и надеюсь, что он является не более чем недоразумением и произошел единократно без какого-либо злого умысла. Естественно, создать условия для того, чтобы все участники могли высказать свои справедливые претензии к проверке работ является нашей целью.
	
	\
	
	\begin{flushright}
		Ответственный за параллель 7 класса, \\
		Капустин А.Д.
	\end{flushright} 
\end{document}