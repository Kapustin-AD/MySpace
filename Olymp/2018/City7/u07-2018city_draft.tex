%\documentclass[11pt,a4paper,onecolumn]{article}

\usepackage{amssymb,amsmath,amsfonts}
\usepackage{bm}
\usepackage[margin = 1.5cm,
			marginparsep = 0.2cm,
			headsep = 0.5cm,
			footskip = 0.7cm, 	
			marginparwidth = 1cm]{geometry}
%\usepackage{showframe}			
			
\usepackage{indentfirst,misccorr}
\usepackage{calc}
\pagestyle{empty}

\setlength{\parindent}{0pt}
\setlength{\parskip}{0pt}

% ШРИФТЫ
\usepackage{ifpdf}
% if pdflatex, then
\ifpdf
	\usepackage[utf8]{inputenc}
	\usepackage[T2A]{fontenc}
	\usepackage[russian]{babel}
	\usepackage[unicode]{hyperref}
	\usepackage[pdftex]{graphicx}
	\usepackage{cmap}
	\usepackage{xfrac}
	%\usepackage[pdftex]{graphicx}
	%\usepackage{cmlgc}
\else % if xelatex, then
	\usepackage{unicode-math}
	%\usepackage{fouriernc}
	% xelatex specific packages
	\usepackage[xetex]{hyperref}
	\usepackage{xltxtra} % \XeLaTeX macro
	\usepackage{xunicode} % some extra unicode support
	\defaultfontfeatures{Mapping = tex-text}
	\usepackage{polyglossia} % instead of babel in xelatex
	\usepackage{indentfirst} %
	\setdefaultlanguage[spelling = modern]{russian}
	\setotherlanguage{english} %% объявляет второй язык документа
	\newfontfamily\cyrillicfont{Linux Libertine}
	\setmainfont{Linux Libertine}
	\setmonofont{Consolas}
	\setsansfont{Linux Biolinum O}
	\setmathfont{Cambria Math}
\fi


% УСЛОВИЯ ЗАДАЧ
\usepackage{tabularx}
\usepackage{makecell}
\usepackage{enumitem}
% счётчик задач
\newcounter{notask}
\setcounter{notask}{1}

% \task{УСЛОВИЕ ЗАДАЧИ}
% задача без картинки
% оформлена как таблица с двумя колонками
% ширина первой колонки (номер столбца) фиксирована, 0.3cm
% ширина второй колонки автоматически рассчитывается из ширины
% страницы (с учётом всевозможных отступов)
\newcommand{\task}[1]{
	\begin{tabularx}{\textwidth}{|c|X|}
		\cline{1-2}
		\makecell*[{{p{0.5cm}}}]{ \centering \arabic{notask}} &
		\makecell*[{{p{\hsize}}}]{ #1 } \\
		\cline{1-2}
	\end{tabularx}
	
	\vspace{-1pt}
	
	\addtocounter{notask}{1}
}


% \taskpic[ШИРИНА КАРТИНКИ]{УСЛОВИЕ ЗАДАЧИ}{КАРТИНКА}
% задача с картинкой
% оформлена как таблица с тремя колонками
% первый аргумент - необязательный, по умолчанию ширина картинки равна
% 4cm, но можно выставить свою
% ширина второй колонки (условие задачи) рассчитывается из ширины
% страницы и ширины картинки
\newcommand{\taskpic}[3][4cm]{
	\begin{tabularx}{\textwidth}{|c|X|c|}
		\cline{1-3}
		\makecell*[{{p{0.5cm}}}]{ \centering \arabic{notask}} &
		\makecell*[{{p{\hsize}}}]{ #2 } &
		\makecell*[{{p{#1}}}]{ \centering #3} \\
		\cline{1-3}
	\end{tabularx}
	
	\vspace{-1pt}
	
	\addtocounter{notask}{1}
}


% Здесь идут команды для tikz если они будут конфликтными надо расскоментировать следующую строчку
%\endinput
\usepackage{tikz}
\usepackage{pgfplots}
\usepackage{wrapfig}
\usepackage{subfig}

% основные библиотеки tikz
\usetikzlibrary{
	arrows,
	calc,
	patterns,
	intersections,
	decorations.pathreplacing,
	decorations.pathmorphing,
	decorations.text,
	decorations.markings,
	shapes,
	positioning
}

% библиотеки для электричества
\usetikzlibrary{circuits.ee, circuits.ee.IEC}

% для работы с графиками            
\pgfplotsset{compat = newest}

% основные используемые стили
% стиль для стрелки
\tikzset{
	>=latex,
	% платформа: пол или потолок  
	interface/.style = {
		postaction = {
			draw,
			decorate,
			decoration = {
				border,
				angle = 45,
				amplitude = 0.2cm,
				segment length = 1mm
			}
		}
	},
	% пружина  
	spring/.style = {
		decorate,
		decoration = {
			coil,
			amplitude = 1mm,
			segment length = 1mm
		},
		thick
	},
	% заряд, вершина, просто точка
	dot/.style = {
		inner sep = 0mm,
		minimum size = 0.18cm,
		fill,
		circle
	},
	% стрелка в середине отрезка  
	marrow/.style = {
		postaction = {
			draw,
			decorate,
			decoration = {
				markings,
				mark = at position 0.6 with {\arrow{latex}}
			}
		}
	}
}


\tikzstyle{dotnode} = [draw, fill, inner sep = 0pt, minimum size = 1mm, circle]

\tikzstyle{termnode} = [draw, fill = white, inner sep = 0pt, minimum size = 1.5mm, circle]

\newcommand{\termcircuit}[2]{
	\node[termnode] at ({#1}, {#2}) {};
	\draw ({#1 + 0.12}, {#2 + 0.15}) -- ({#1 - 0.12}, {#2 - 0.15});
}




% обозначение угла
\tikzset{
	arcnode/.style = {
		decoration = {
			markings,
			raise = 2mm,
			mark = at position 0.5 with {
				\node[inner sep = 0] {#1};
			}
		},
		postaction = {decorate}
	}
}

% команда для отметки угла: проводит дугу между двумя лучами,
% проведёнными между точками #2--#3 и #2--#4
% первый аргумент - необязательный, стиль линии
% #2,#3,#4 - точки
% #5 - радиус дуги для обозначения угла
% #6 - обозначение угла, например, $\alpha$
\newcommand*\marktheangle[6][]{
	\draw[thick, arcnode = {#6}, #1] let \p2 = ($(#3)-(#2)$),
	\p3 = ($(#4)-(#2)$),
	\n2 = {atan2(\x2,\y2)},
	\n3 = {atan2(\x3,\y3)}%
	in ($(\n2:#5)+(#2)$) arc (\n2:\n3:#5);
}


% амперметр
\tikzset{circuit declare symbol = ammeter}
\tikzset{
	set ammeter graphic = {
		draw,
		generic circle IEC,
		minimum size = 5mm,
		info=center:A
	}
} 

% вольтметр
\tikzset{circuit declare symbol = voltmeter}
\tikzset{
	set voltmeter graphic = {
		draw,
		generic circle IEC,
		minimum size = 5mm,
		info=center:V
	}
} 

% кружок
\tikzset{circuit declare symbol = meter}
\tikzset{
	set meter graphic = {
		draw,
		generic circle IEC,
		minimum size = 5mm
	}
} 


\newcommand{\platform}[2]{
	\draw [draw = none, pattern = north west lines] 
	#1 -- ( $#1!0.15cm!90:#2$ ) -- ( $#2!0.15cm!-90:#1$ ) -- #2;
	\draw #1 -- #2
}


\documentclass[11pt,a4paper,onecolumn]{article}
\usepackage{amssymb,amsmath,amsfonts}
\usepackage{bm}
\usepackage[margin = 1.5cm,
		    marginparsep = 0.2cm,
			headsep = 0.5cm,
			footskip = 0.7cm, 	
			marginparwidth = 1cm]{geometry}
%\usepackage{showframe}			

\usepackage{indentfirst}
\usepackage{calc}
\pagestyle{empty}
\usepackage{physsummer}

\setlength{\parindent}{0pt}
\setlength{\parskip}{0pt}

%% Рисование районного тура с двумя шапками:
%% Следует использовать так:
%% 		\OlympSetReg{ год в формате 16/17 }{ класс }{ номер варианта }{ условие }
\newcommand{\OlympSetRegRussia}[4]{  %% Районный тур всероссийской
	\setcounter{notask}{1}
	\begin{center}
		\textsc{ Всероссийская олимпиада школьников по физике 20#1 г. } \\
		\textsc{ Районный этап } \\
		\textit{ Решения см. на сайте { \underline{www.physolymp.spb.ru} } } \\
	\end{center}
	\vspace{ -1cm }
	\parbox{ 0.5\textwidth }{ \flushleft  \textsc{#2 класс} }
	\parbox{ 0.5\textwidth }{ \flushright \textsc{#3-й вариант } } \\[ 0.1cm ]
	#4
	\vfill
	\begin{center}
		\textsc{Оставьте условие себе!}
	\end{center}
	\clearpage
}

\newcommand{\OlympSetRegSPb}[4]{ 	
	\setcounter{notask}{1}
	\begin{center}
		\textsc{ Городская открытая олимпиада школьников по физике  20#1 г. } \\
		\textsc{ Отборочный этап } \\
		%\textsc{ I городской тур } \\
		\textit{ Решения см. на сайте { \underline{www.physolymp.spb.ru} } } \\
	\end{center}
	\vspace{ -1cm }
	\parbox{ 0.5\textwidth }{ \flushleft  \textsc{#2 класс } }
	\parbox{ 0.5\textwidth }{ \flushright \textsc{#3-й вариант } }\\[ 0.1cm ]
	#4
	\vfill
	\begin{center}
		\textsc{Оставьте условие себе!}
	\end{center}
	\clearpage	
}


%% Рисование городского тура с выводом

\newcommand{\OlympHeader}[4]{
	\setcounter{notask}{1}
	\begin{center}
		\textsc{  Городская открытая олимпиада школьников по физике 20#1 г. } \\
		%\textsc{ Основной этап } \\
		\textsc{ Теоретический тур } \\
		\textit{ Решения см. на сайте { \underline{www.physolymp.spb.ru} } } \\
	\end{center}
	\vspace{ -1cm }
	\parbox{ 0.5\textwidth }{ \flushleft  \textsc{#2 класс} }
	\parbox{ 0.5\textwidth }{ \flushright  \textsc{Первый этап} }
		#3
	\vfill
	\begin{center}
		\textsc{Оставьте условие себе!}
	\end{center}
	\clearpage
	
	\begin{center}
		\textsc{  Городская открытая олимпиада школьников по физике 20#1 г. } \\
		%\textsc{ Основной этап } \\
		\textsc{ Теоретический тур } \\
		\textit{ Решения см. на сайте { \underline{www.physolymp.spb.ru} } } \\
	\end{center}
	\vspace{ -1cm }
	\parbox{ 0.5\textwidth }{ \flushleft  \textsc{8 класс} }
	\parbox{ 0.5\textwidth }{ \flushright  \textsc{Второй этап} }
		#4
	\vfill
	\begin{center}
		\textsc{Оставьте условие себе!}
	\end{center}
	\clearpage	
}

\newcommand{\gecs}[2]{(#1-0.5,#2+0.85) -- ++(1, 0) -- ++(0.5, -0.85) -- ++(-0.5, -0.85) -- ++(-1, 0) -- ++(-0.5, 0.85) -- cycle}
\newcommand{\OlympHeaderG}[1]{
	%\setcounter{notask}{1}
	\begin{center}
		\text{ Городская открытая олимпиада школьников по физике  2017/18 г. } \\
		\text{ Теоретический тур } \\
		%\textsc{ I городской тур } \\
		\textit{ Решения см. на сайте { \underline{www.physolymp.spb.ru} } } \\
		% \texttt{ Версия от \today }
	\end{center}
	\vspace{ -0.7cm }
	\parbox{ 0.5\textwidth }{ \flushleft  \text{7 класс} }
	\parbox{ 0.5\textwidth }{ \flushright \text{#1 этап} }
}
\setcounter{notask}{1}

\usepackage{libertine}
\begin{document}

\OlympHeaderG{Первый}
 
\task { %	Некому члену жюри поручили проверить первую задачу районного этапа олимпиады по физике у всего седьмого класса. Работ очень много, и в каждой пятой работе первая задача не написана. В таком случае проверяющий ставит прочерк в таблице и не тратит времени на проверку. За остальные работы он ставит либо 0 баллов (время на проверку --- 20 c), либо 10 (время на проверку --- 100 с). Он проверил $1000$ работ, затратив в среднем $64$~секунды на работу. Какая доля чернил останется в ручке проверяющего, если она расчитана на 100 м непрерывной чернильной линии? Расход чернил на различные отметки проверяющего указан на рисунке, сторона клетки равна $0{,}5$~см.
	Василий проверяет первую задачу районного этапа олимпиады по физике во всех работах седьмого класса. Оказалось, что в каждой пятой работе первая задача не написана. В таком случае проверяющий ставит прочерк в таблице результатов (см. рис.), не тратя времени на проверку. В остальных работах он ставит либо $0$ баллов (время проверки --- $20$~c), либо $10$ (время проверки --- $100$~с). Проверив $1000$ работ, Василий затратил в среднем $64$~секунды на работу. Какая доля чернил останется в ручке проверяющего, если она рассчитана на $100$~м непрерывной чернильной линии? Василий всегда одинаковым образом ставит прочерки, $0$ и $10$, пример отметок приведен на рисунке, сторона клетки $0{,}5$~см.}
\task { 	У Пети была пятидесятилитровая бочка. Он поместил в нее максимально возможное количество каменных шаров радиусом $10$~см. Оказалось, что если залить в бочку с шарами 13 литров воды, то уровень воды совпадёт с краем бочки. Сколько воды выльется, если после этого добавить в бочку с большими камнями максимально возможное количество круглых каменных шариков радиусом  $1$~мм?}
\task { 	В одной древней цивилизации существовал обычай, по которому любой человек мог получить себе такой квадратный участок земли, который он сможет обежать за сутки. Один из них рассчитал маршрут так, чтобы добежать ровно за $24$~часа. Однако, через $3$~часа он понял, что переоценил свои силы и оставшееся время сможет бежать только с вдвое меньшей скоростью. Не растерявшись, он быстро перестроил маршрут и добежал вовремя. Найдите отношение площади участка, который он получил, к площади участка, который он хотел получить изначально.  }
\task { 	Для обогрева в течение зимы в печке сжигались дрова. Известно, что в силу климатических условий расход дров с каждым днем не уменьшался. В некоторые дни проводились измерения расхода дров за день, результаты которых приведены на графике (см. рис.). Оцените максимальное и минимальное количество кубометров дров, которое могло быть сожжено за $90$~дней.}
\begin{figure}[h!]
	\begin{minipage}{0.49\linewidth}
		\centering
		\begin{tikzpicture}[scale=1,
	/pgfplots/axis labels at tip/.style={ 
		xlabel style={ at={ (current axis.right of origin) }, 
			yshift = 1 ex,
			anchor = south east,
			fill=white}, 
		ylabel style={ at={ (current axis.above origin) }, 
			xshift = 1 ex, 
			anchor = north west,
			fill = white} } ]
	\begin{axis}[
	grid style={line width=.1pt, draw=gray!80},
		line width = 2pt, 
		axis x line = middle,
		axis y line = middle,
		axis labels at tip,
		xmin = 0, xmax = 95,
		ymin = 0, ymax = 0.55,
		xtick = {0, 10, ..., 100},
		ytick = {0, 0.1, ..., 0.6},
		xlabel = {$t,\text{ дней }$},
		ylabel = {$Q,\text{ м}^3\text{/день}$},
		grid = both,
		major grid style={line width = 1.3pt,draw=black!50},
		minor grid style={line width=.4pt, draw=black!50},
		major tick length = 7pt,
	    every major tick/.style={
			black,
	        line width = 2pt,
        },
		minor tick length = 4pt,
	    every minor tick/.style={
			black,
	        line width = 1pt,
        },
		minor tick num = 0]
	\fill[blue] (20, 0.1) circle(1mm); 
	\fill[blue] (50, 0.2) circle(1mm); 
	\fill[blue] (60, 0.3) circle(1mm); 
	\fill[blue] (90, 0.4) circle(1mm); 
 
		
	\end{axis}
\end{tikzpicture}
		\caption*{Рисунок к задаче 4}
	\end{minipage}	
	\hfill
	\begin{minipage}{0.49\linewidth}
		\centering
		\begin{tikzpicture}[scale = 1.5]
	\draw[step=0.5,black,thin] (0, 0) grid (4, -3); 
	\draw [<->] (0.5, 0.1) -- (1, 0.1)node[midway, above]{$0{,}5~\text{см}$};
%	\draw[dashed] (1, 0) -- (1, 0.4);
%	\draw[dashed] (0.5, 0) -- (0.5, 0.4);
	\draw[ultra thick] (0.5, -0.5) -- (1.5, -0.5) -- (1.5, -1.5) -- (0.5, -1.5) -- cycle;
	\draw[ultra thick] (0.5, -1.5) -- (1.5, -1.5) -- (1.5, -2.5) -- (0.5, -2.5) -- cycle;
	\draw[ultra thick] (1.5, -0.5) -- (2.5, -0.5) -- (2.5, -1.5) -- (1.5, -1.5) -- cycle;
	\draw[ultra thick] (1.5, -1.5) -- (2.5, -1.5) -- (2.5, -2.5) -- (1.5, -2.5) -- cycle;
	\draw[ultra thick] (2.5, -0.5) -- (3.5, -0.5) -- (3.5, -1.5) -- (2.5, -1.5) -- cycle;
	\draw[ultra thick] (2.5, -1.5) -- (3.5, -1.5) -- (3.5, -2.5) -- (2.5, -2.5) -- cycle;
	\draw (1, -1)node{\Huge{$1$}};
	\draw (2, -1)node{\Huge{$2$}};
	\draw (3, -1)node{\Huge{$3$}};	
	\draw[line width = 3pt, red] (0.75, -2) -- (1.25, -2);
	\draw[line width = 3pt, red] (1.75, -1.75) -- (2.25, -1.75) -- (2.25, -2.25) -- (1.75, -2.25) -- cycle;
	\draw[line width = 3pt, red] (2.75, -1.75) -- (2.75, -2.25);
	\draw[line width = 3pt, red] (3, -1.75) -- (3.5, -1.75) -- (3.5, -2.25) -- (3, -2.25) -- cycle;
\end{tikzpicture}
		\caption*{Рисунок к задаче 3}
	\end{minipage}
\end{figure}

\begin{center}
	\text{Оставьте условие себе!}
\end{center}

\newpage


\OlympHeaderG{Второй}

\task {Миша измерял скорость улитки, ползущей вдоль веревки, привязанной к столбу (см. рис.). Каждые $100$~секунд он отмечал положение улитки, очень быстро завязывая рядом с ней узелок. В какой-то момент ему стало скучно, и он стал завязывать по два узелка в одном месте, вместо одного. Вернувшись домой с веревкой, он измерил расстояния между узелками (не развязывая их) и, поделив их на $100$~секунд, вычислил скорости движения улитки. Если брать расстояние между одинарными узелками, скорость оказалась равна $1{,}3$~мм/с, а если между двойными --- $1{,}1$~мм/с. Найдите скорость движения улитки, если на протяжении всего пути она оставалась постоянной. Все узелки одинаковые. }
\taskpic { 	Имеется сосуд, заполненный водой и закрытый двумя поршнями площадью $S_1 = 150~\text{см}^2$ и $S_2 = 75~\text{см}^2$, верхний из которых очень легкий, а нижний имеет массу $m = 1$~кг. Поршни связаны пружиной с жесткостью $k = 50~\text{Н/м}$. Найдите удлинение пружины в положении равновесия системы, если в нерастянутом состоянии ее длина пренебрежимо мала. Атмосферное давление $p_0 = 100$~кПа.}{\begin{tikzpicture}[scale = 0.8]
	\draw[thick] (0, 0.5) -- (0, -1) -- (-3, -1) -- (-3, 3) -- (-0.15, 3) -- (-0.15, 1.5);
	\draw[thick] (-1, 0.5) -- (-1, -0.5) -- (-2.5, -0.5) -- (-2.5, 2.5) -- (-0.85, 2.5) -- (-0.85, 1.5);
	\fill[blue, opacity = 0.2] (0, -0.2) -- (0, -1) -- (-3, -1) -- (-3, 3) -- (-0.15, 3) -- (-0.15, 2) -- (-0.85, 2) -- (-0.85, 2.5) -- (-2.5, 2.5) -- (-2.5, -0.5) -- (-1, -0.5) -- (-1, -0.2) -- cycle;
	\draw[pattern = north west lines] (-0.83, 2) rectangle +(0.66, -0.15);
	\draw[pattern = north west lines] (-0.98, 0)node[left]{$m$} rectangle +(0.96, -0.2);
	\draw[spring, segment length=1.5mm] (-0.5, 1.85) -- (-0.5, 0)node[midway, left]{$k$};
	\draw[<->] (0.8, 0) -- (0.8, 1.85)node[midway, right]{$l$};
	\draw (-0.5, -0.6)node{$S_1$};
	\draw (-0.5, 2.5)node{$S_2$};	
	\draw (1, 2.5)node{$p_0$};
	\draw[dashed] (-0.15, 1.85) -- (0.8, 1.85);
	\draw[dashed] (0, 0) -- (0.8, 0);
\end{tikzpicture}}
\task {	Вблизи химзавода в городе Черноснежинске прошел двухдневный снегопад. За первый день на некоторой ровной площадке выпало $20$~см снега. За второй день выпало еще $25$~кг снега на каждый квадратный метр этой площадки. Найдите глубину снежного покрова в конце второго дня. Считайте, что в покрове любой глубины плотность снега у поверхности нулевая и равномерно увеличивается на $10~\text{кг/м}^3$ за $1$~см.}
\begin{figure}[h!]
	\centering
	\begin{tikzpicture}[scale = 1.5]
	\draw[thick] (0, 0) -- (1, 0) .. controls (0.7, 0.5) and (1.3, 0.5) .. (1, 0) -- (3.7, 0) arc (270:330:0.35);
	\draw[ultra thick] (4, 0) -- (4, 1);
	\draw[thick] (0.5, 0) .. controls (0.2, 0.5) and (0.8, 0.5) .. (0.5, 0);
	\draw[thick, rotate around={15:(1.4, 0)}] (1.4, 0) .. controls (1.1, 0.5) and (1.7, 0.5) .. (1.4, 0);
	\draw[thick, rotate around={-15:(1.4, 0)}] (1.4, 0) .. controls (1.1, 0.5) and (1.7, 0.5) .. (1.4, 0);
	\draw[thick] (3.5, 1.5) rectangle +(1, -0.5);
	\draw (4, 1.25)node{столб};
	\draw[dashed] (-0.5, -0.4) -- (1.4, -0.4);
	\fill (1.4, -0.3) circle (0.1);
	\draw (1.4, -0.4) arc(270:360:0.2);
	\draw (1.4, -0.4) arc(270:330:0.4);
\end{tikzpicture}
	\caption*{Рисунок к задаче 5}
\end{figure}
\begin{center}
	\text{Оставьте условие себе!}
\end{center}
\clearpage




\end{document}
