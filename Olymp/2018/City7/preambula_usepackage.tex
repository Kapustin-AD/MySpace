\documentclass[11pt,a4paper,onecolumn]{article}
\usepackage{amssymb,amsmath,amsfonts}
\usepackage{bm}
\usepackage[margin = 1.5cm,
		    marginparsep = 0.2cm,
			headsep = 0.5cm,
			footskip = 0.7cm, 	
			marginparwidth = 1cm]{geometry}
%\usepackage{showframe}			

\usepackage{indentfirst}
\usepackage{calc}
\pagestyle{empty}
\usepackage{physsummer}

\setlength{\parindent}{0pt}
\setlength{\parskip}{0pt}

%% Рисование районного тура с двумя шапками:
%% Следует использовать так:
%% 		\OlympSetReg{ год в формате 16/17 }{ класс }{ номер варианта }{ условие }
\newcommand{\OlympSetRegRussia}[4]{  %% Районный тур всероссийской
	\setcounter{notask}{1}
	\begin{center}
		\textsc{ Всероссийская олимпиада школьников по физике 20#1 г. } \\
		\textsc{ Районный этап } \\
		\textit{ Решения см. на сайте { \underline{www.physolymp.spb.ru} } } \\
	\end{center}
	\vspace{ -1cm }
	\parbox{ 0.5\textwidth }{ \flushleft  \textsc{#2 класс} }
	\parbox{ 0.5\textwidth }{ \flushright \textsc{#3-й вариант } } \\[ 0.1cm ]
	#4
	\vfill
	\begin{center}
		\textsc{Оставьте условие себе!}
	\end{center}
	\clearpage
}

\newcommand{\OlympSetRegSPb}[4]{ 	
	\setcounter{notask}{1}
	\begin{center}
		\textsc{ Городская открытая олимпиада школьников по физике  20#1 г. } \\
		\textsc{ Отборочный этап } \\
		%\textsc{ I городской тур } \\
		\textit{ Решения см. на сайте { \underline{www.physolymp.spb.ru} } } \\
	\end{center}
	\vspace{ -1cm }
	\parbox{ 0.5\textwidth }{ \flushleft  \textsc{#2 класс } }
	\parbox{ 0.5\textwidth }{ \flushright \textsc{#3-й вариант } }\\[ 0.1cm ]
	#4
	\vfill
	\begin{center}
		\textsc{Оставьте условие себе!}
	\end{center}
	\clearpage	
}


%% Рисование городского тура с выводом

\newcommand{\OlympHeader}[4]{
	\setcounter{notask}{1}
	\begin{center}
		\textsc{  Городская открытая олимпиада школьников по физике 20#1 г. } \\
		%\textsc{ Основной этап } \\
		\textsc{ Теоретический тур } \\
		\textit{ Решения см. на сайте { \underline{www.physolymp.spb.ru} } } \\
	\end{center}
	\vspace{ -1cm }
	\parbox{ 0.5\textwidth }{ \flushleft  \textsc{#2 класс} }
	\parbox{ 0.5\textwidth }{ \flushright  \textsc{Первый этап} }
		#3
	\vfill
	\begin{center}
		\textsc{Оставьте условие себе!}
	\end{center}
	\clearpage
	
	\begin{center}
		\textsc{  Городская открытая олимпиада школьников по физике 20#1 г. } \\
		%\textsc{ Основной этап } \\
		\textsc{ Теоретический тур } \\
		\textit{ Решения см. на сайте { \underline{www.physolymp.spb.ru} } } \\
	\end{center}
	\vspace{ -1cm }
	\parbox{ 0.5\textwidth }{ \flushleft  \textsc{8 класс} }
	\parbox{ 0.5\textwidth }{ \flushright  \textsc{Второй этап} }
		#4
	\vfill
	\begin{center}
		\textsc{Оставьте условие себе!}
	\end{center}
	\clearpage	
}
