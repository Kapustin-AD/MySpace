	Поймем, как среднее время $T$, затрачиваемое на проверку одной работы, связано с долей нулевых работ $\alpha$. Время $T$ запишется как
\[
	T = \frac{1}{5} \cdot 0~\text{с} + \alpha \cdot20~\text{с} + \left( \frac{4}{5} - \alpha \right) \cdot 100~\text{с}.
\]
Нас интересует расход чернил, поэтому по аналогии со средним временем найдем величину среднего расхода чернил на одну работу. Эта величина определяется следующим выражением
\[
	P = \frac{1}{5} \cdot 0{,}5~\text{см} + \alpha \cdot 2~\text{см} + \left( \frac{4}{5} - \alpha \right) \cdot 2{,}5~\text{см}.
\]
Выразим $\alpha$ через $T$ из первого уравнения
\[
	\alpha = \frac{\left( 80 - T \right)}{80} = \frac{1}{5},
\]
\[
	P = 2{,}1~\text{см} - \frac{1}{5} \cdot 0{,}5~\text{см} = 2~\text{см}.
\]

	Теперь, чтобы найти расход чернил нужно умножить эту величину на количество проверенных работ. Расход окажется равным
\[
	L = 2000~\text{см} = 20~\text{м}.
\]
Тогда в ручке останется $80~\text{м}$, что составляет $0{,}8$ от первоначального количества.

\underline{\textbf{Ответ:}} $0{,}8$.