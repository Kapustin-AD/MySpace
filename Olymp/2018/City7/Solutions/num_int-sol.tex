\begin{figure}[h!]
	\begin{minipage}{0.49\linewidth}
	\centering
	\begin{tikzpicture}[scale=1,
	/pgfplots/axis labels at tip/.style={ 
		xlabel style={ at={ (current axis.right of origin) }, 
			yshift = 1 ex,
			anchor = south east,
			fill=white}, 
		ylabel style={ at={ (current axis.above origin) }, 
			xshift = 1 ex, 
			anchor = north west,
			fill = white} } ]
	\begin{axis}[
	grid style={line width=.1pt, draw=gray!80},
		line width = 2pt, 
		axis x line = middle,
		axis y line = middle,
		axis labels at tip,
		xmin = 0, xmax = 95,
		ymin = 0, ymax = 0.55,
		xtick = {0, 10, ..., 100},
		ytick = {0, 0.1, ..., 0.6},
		xlabel = {$t,\text{ дней }$},
		ylabel = {$Q,\text{ м}^3\text{/день}$},
		grid = both,
		major grid style={line width = 1.3pt,draw=black!50},
		minor grid style={line width=.4pt, draw=black!50},
		major tick length = 7pt,
	    every major tick/.style={
			black,
	        line width = 2pt,
        },
		minor tick length = 4pt,
	    every minor tick/.style={
			black,
	        line width = 1pt,
        },
		minor tick num = 0]
	\fill[blue] (20, 0.1) circle(1mm); 
	\fill[blue] (50, 0.2) circle(1mm); 
	\fill[blue] (60, 0.3) circle(1mm); 
	\fill[blue] (90, 0.4) circle(1mm); 
 	\draw[blue, line width = 1pt] (0, 0) to[in=180, out=30] ( 20, 0.1) to[in = 270, out = 0] (50, 0.2) to[in = 200, out = 90] (60, 0.3) to[in = 240,out = 20] (90, 0.4);
		
	\end{axis}
	\end{tikzpicture}
	\caption{Пример функции расхода дров от времени.}
	\end{minipage}
	\begin{minipage}{0.49\linewidth}
	\centering
	\begin{tikzpicture}[scale=1,
	/pgfplots/axis labels at tip/.style={ 
		xlabel style={ at={ (current axis.right of origin) }, 
			yshift = 1 ex,
			anchor = south east,
			fill=white}, 
		ylabel style={ at={ (current axis.above origin) }, 
			xshift = 1 ex, 
			anchor = north west,
			fill = white} } ]
	\begin{axis}[
	grid style={line width=.1pt, draw=gray!80},
		line width = 2pt, 
		axis x line = middle,
		axis y line = middle,
		axis labels at tip,
		xmin = 0, xmax = 95,
		ymin = 0, ymax = 0.55,
		xtick = {0, 10, ..., 100},
		ytick = {0, 0.1, ..., 0.6},
		xlabel = {$t,\text{ дней }$},
		ylabel = {$Q,\text{ м}^3\text{/день}$},
		grid = both,
		major grid style={line width = 1.3pt,draw=black!50},
		minor grid style={line width=.4pt, draw=black!50},
		major tick length = 7pt,
	    every major tick/.style={
			black,
	        line width = 2pt,
        },
		minor tick length = 4pt,
	    every minor tick/.style={
			black,
	        line width = 1pt,
        },
		minor tick num = 0]
	\fill[blue] (20, 0.1) circle(1mm); 
	\fill[blue] (50, 0.2) circle(1mm); 
	\fill[blue] (60, 0.3) circle(1mm); 
	\fill[blue] (90, 0.4) circle(1mm); 
 	\draw[blue, line width = 1pt] (0, 0) to[in=180, out=30] ( 20, 0.1) to[in = 270, out = 0] (50, 0.2) to[in = 200, out = 90] (60, 0.3) to[in = 240,out = 20] (90, 0.4);	
 	\fill[blue!80!, opacity = 0.2] (0, 0) rectangle +(20, 0.1);
 	\fill[blue!80!, opacity = 0.2] (20, 0.1) rectangle +(30, 0.1);
 	\fill[blue!80!, opacity = 0.2] (50, 0.2) rectangle +(10, 0.1);
 	\fill[blue!80!, opacity = 0.2] (60, 0.3) rectangle +(30, 0.1);
	\end{axis}
	\end{tikzpicture}
	\caption{Области, в которых может лежать линия графика.}
	\end{minipage}
\end{figure}		

	По условию, расход дров с каждым днем не убывал. Это значит, что он задается линией на графике, проходящей через заданные точки, и, всегда направленной вправо и верх. Тогда можо провести горизональные и вертикальные линии, проходящие через известные нам точки графика, и сам график будет лежать внутри получившихся прямоугольников. 
	
	Если бы мы знали всю линию графика, мы могли бы точно найти расход дров как площадь фигуры, ограниченной этим графиком. Чтобы это понять, можно усмотреть явную аналогию с графиком скорости от времени. Но раз нам известны только некоторые условия, задающие целый набор возможных линий расхода дров, нужно найти максимальное и минмальное возможные значения расхода дров. Для этого из набора всех возможных линий следует выбрать те две, которые задают минимальную и максимальную площади. Ясно, что эти линии являются границами нарисованных прямоугольников.

\begin{figure}[h!]
	\begin{minipage}{0.49\linewidth}
	\centering
	\begin{tikzpicture}[scale=1,
	/pgfplots/axis labels at tip/.style={ 
		xlabel style={ at={ (current axis.right of origin) }, 
			yshift = 1 ex,
			anchor = south east,
			fill=white}, 
		ylabel style={ at={ (current axis.above origin) }, 
			xshift = 1 ex, 
			anchor = north west,
			fill = white} } ]
	\begin{axis}[
	grid style={line width=.1pt, draw=gray!80},
		line width = 2pt, 
		axis x line = middle,
		axis y line = middle,
		axis labels at tip,
		xmin = 0, xmax = 95,
		ymin = 0, ymax = 0.55,
		xtick = {0, 10, ..., 100},
		ytick = {0, 0.1, ..., 0.6},
		xlabel = {$t,\text{ дней }$},
		ylabel = {$Q,\text{ м}^3\text{/день}$},
		grid = both,
		major grid style={line width = 1.3pt,draw=black!50},
		minor grid style={line width=.4pt, draw=black!50},
		major tick length = 7pt,
	    every major tick/.style={
			black,
	        line width = 2pt,
        },
		minor tick length = 4pt,
	    every minor tick/.style={
			black,
	        line width = 1pt,
        },
		minor tick num = 0]
	\draw[red] (0, 0) to (0, 0.1) to (20, 0.1) to (20, 0.2) to (50, 0.2) to (50, 0.3) to (60, 0.3) to (60, 0.4) to (90, 0.4);
	\fill[blue] (20, 0.1) circle(1mm); 
	\fill[blue] (50, 0.2) circle(1mm); 
	\fill[blue] (60, 0.3) circle(1mm); 
	\fill[blue] (90, 0.4) circle(1mm); 	
	\end{axis}
	\end{tikzpicture}
	\caption{Линия, задающая максимальную площадь.}
	\end{minipage}
	\begin{minipage}{0.49\linewidth}
	\centering
	\begin{tikzpicture}[scale=1,
	/pgfplots/axis labels at tip/.style={ 
		xlabel style={ at={ (current axis.right of origin) }, 
			yshift = 1 ex,
			anchor = south east,
			fill=white}, 
		ylabel style={ at={ (current axis.above origin) }, 
			xshift = 1 ex, 
			anchor = north west,
			fill = white} } ]
	\begin{axis}[
	grid style={line width=.1pt, draw=gray!80},
		line width = 2pt, 
		axis x line = middle,
		axis y line = middle,
		axis labels at tip,
		xmin = 0, xmax = 95,
		ymin = 0, ymax = 0.55,
		xtick = {0, 10, ..., 100},
		ytick = {0, 0.1, ..., 0.6},
		xlabel = {$t,\text{ дней }$},
		ylabel = {$Q,\text{ м}^3\text{/день}$},
		grid = both,
		major grid style={line width = 1.3pt,draw=black!50},
		minor grid style={line width=.4pt, draw=black!50},
		major tick length = 7pt,
	    every major tick/.style={
			black,
	        line width = 2pt,
        },
		minor tick length = 4pt,
	    every minor tick/.style={
			black,
	        line width = 1pt,
        },
		minor tick num = 0]
	\draw[red] (0, 0) to (20, 0) to (20, 0.1) to (50, 0.1) to (50, 0.2) to (60, 0.2) to (60, 0.3) to (90, 0.3) to (90, 0.4);
	\fill[blue] (20, 0.1) circle(1mm); 
	\fill[blue] (50, 0.2) circle(1mm); 
	\fill[blue] (60, 0.3) circle(1mm); 
	\fill[blue] (90, 0.4) circle(1mm); 
	\end{axis}
	\end{tikzpicture}
	\caption{Линия, задающая минимальную площадь.}
	\end{minipage}
\end{figure}

	Теперь для получения ответа осталось только посчитать получившиеся площади. В случае максимального расхода 
\[
	P_{max} = \left( 0{,}1 \cdot 20 + 0{,}2 \cdot 30 + 0{,}3 \cdot 10 + 0{,}4 \cdot 30 \right)~\text{м}^3 = 23~\text{м}^3. 
\]
В случае минимального
\[
	P_{min} = \left( 0{,}1 \cdot 30 + 0{,}2 \cdot 10 + 0{,}3 \cdot 30 + 0{,}4 \cdot 1\right)~\text{м}^3 = 14{,}4~\text{м}^3. 
\]

\underline{\textbf{Ответ:}} $23~\text{м}^3$ и $14{,}4~\text{м}^3$.