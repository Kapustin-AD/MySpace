\begin{figure}[h!]
\centering
	\begin{tikzpicture}[scale = 1.7]
	\draw[thick] (0, 0.5) -- (0, -1) -- (-3, -1) -- (-3, 3) -- (-0.15, 3) -- (-0.15, 1.5);
	\draw[thick] (-1, 0.5) -- (-1, -0.5) -- (-2.5, -0.5) -- (-2.5, 2.5) -- (-0.85, 2.5) -- (-0.85, 1.5);
	\fill[blue, opacity = 0.2] (0, -0.2) -- (0, -1) -- (-3, -1) -- (-3, 3) -- (-0.15, 3) -- (-0.15, 2) -- (-0.85, 2) -- (-0.85, 2.5) -- (-2.5, 2.5) -- (-2.5, -0.5) -- (-1, -0.5) -- (-1, -0.2) -- cycle;
	\draw[pattern = north west lines] (-0.83, 2) rectangle +(0.66, -0.15);
	\draw[pattern = north west lines] (-0.98, 0)node[left]{$m$} rectangle +(0.96, -0.2);
	\draw[spring, segment length=1.5mm] (-0.5, 1.85) -- (-0.5, 0)node[midway, left]{$k$};
	\draw[<->] (0.8, 0) -- (0.8, 1.85)node[midway, right]{$l$};
	\draw (-0.5, -0.6)node{$S_1$};
	\draw (-0.5, 2.5)node{$S_2$};	
	\draw (1, 2.5)node{$p_0$};
	\draw[dashed] (-0.15, 1.85) -- (0.8, 1.85);
	\draw[dashed] (0, 0) -- (0.8, 0);
	\end{tikzpicture}
\end{figure}	

Рассмотрим условия, выполнение которых необходимо для равновесия системы. Чтобы система могла оставаться в покое, сумма всех сил, действующих на каждый элемент, должна быть равна нулю. Найдем все силы, действующие на верхний и нижний поршни.

\begin{figure}[h!]
	\begin{minipage}{0.49\linewidth}
		\centering
		\begin{tikzpicture}[scale = 2.5]
			\clip (-0.5, 2) circle(0.9cm);
			\draw[thick] (0, 0.5) -- (0, -1) -- (-3, -1) -- (-3, 3) -- (-0.15, 3) -- (-0.15, 1.5);
			\draw[thick] (-1, 0.5) -- (-1, -0.5) -- (-2.5, -0.5) -- (-2.5, 2.5) -- (-0.85, 2.5) -- (-0.85, 1.5);
			\draw[pattern = north west lines] (-0.83, 2) rectangle +(0.66, -0.15);
			\draw[pattern = north west lines] (-0.98, 0)node[left]{$m$} rectangle +(0.96, -0.2);
			\draw[spring, segment length=1.5mm] (-0.5, 1.85) -- (-0.5, 0)node[midway, left]{$k$};
			\draw[<->] (0.5, 0) -- (0.5, 1.85)node[midway, right]{$l$};
			\draw (-0.5, -0.6);
			\draw (-0.5, 2.5);	
			\draw (1, 2.5)node{$p_0$};
			\draw[red, ultra thick, ->] (-0.4, 1.925) -- (-0.4, 2.8)node[midway, right]{$F_{\textit{дав}}$}; 
			\draw[blue, ultra thick, ->] (-0.6, 1.925) -- (-0.6, 1.075)node[midway, left]{$F_{\textit{упр}}$};
		\end{tikzpicture}
		\caption{Верхний поршень}
	\end{minipage}
	\begin{minipage}{0.49\linewidth}
		\centering
		\begin{tikzpicture}[scale = 2.5]
			\clip (-0.5, -0.13) circle(1cm);
			\draw[thick] (0, 0.5) -- (0, -1) -- (-3, -1) -- (-3, 3) -- (-0.15, 3) -- (-0.15, 1.5);
			\draw[thick] (-1, 0.5) -- (-1, -0.5) -- (-2.5, -0.5) -- (-2.5, 2.5) -- (-0.85, 2.5) -- (-0.85, 1.5);
			\draw[pattern = north west lines] (-0.83, 2) rectangle +(0.66, -0.15);
			\draw[pattern = north west lines] (-0.98, 0)node[left]{$m$} rectangle +(0.96, -0.2);
			\draw[spring, segment length=1.5mm] (-0.5, 1.85) -- (-0.5, 0)node[midway, left]{$k$};
			\draw[<->] (0.5, 0) -- (0.5, 1.85)node[midway, right]{$l$};
			\draw (-0.5, -0.6);
			\draw (-0.5, 2.5);	
			\draw (1, 2.5)node{$p_0$};
			\draw[red, ultra thick, ->] (-0.4, -0.1) -- (-0.4, -0.6)node[midway, right]{$F_{\textit{дав}}$}; 
			\draw[blue, ultra thick, ->] (-0.6, -0.1) -- (-0.6, 0.775)node[midway, left]{$F_{\textit{упр}}$};
			\draw[black, ultra thick, ->] (-0.6, -0.1) -- (-0.6, -0.475)node[midway, left]{$F_{\textit{тяж}}$};
		\end{tikzpicture}
		\caption{Нижний поршень}
	\end{minipage}
\end{figure}

В случае верхнего поршня присутствуют только сила упругости пружины $F_{\textit{упр}}$ и сила давления $F_{\textit{дав}}$. По закону Гука, сила упругости равна 
\[
	F_{\textit{упр}} = k \cdot l,
\]
где $l$ --- удлинение пружины, которое равно расстоянию между поршнями. Сила давления будет равна разности давлений с нижней и верхней сторон поршня, умноженной на площадь поверхности поршня
\[
	F_{\textit{дав}} = \left( p_0 - p \right) \cdot S_2,
\] 
где за $p$ мы обозначили давление воды на уровне верхнего поршня. Условие равновесия верхнего поршня тогда запишется в виде
\begin{equation}
\label{eq72-1}
	k \cdot l = \left( p_0 - p \right) \cdot S_2.
\end{equation}

В случае нижнего поршня присутствует еще сила тяжести, равная
\[
	F_{\textit{тяж}} = mg,
\]
и сила давления выразится через уже введенные величины как
\[
	F_{\textit{дав}} = \left( p_0 - \left( p + \rho g l \right) \right) \cdot S_1,
\]
где $p + \rho g l$ --- давление под нижним поршнем. Оно отличается от давления $p$ на давление столба жидкости высотой $l$. Условие равновесия нижнего поршня тогда запишется в виде
\begin{equation}
\label{eq72-2}
	k \cdot l = \left( p_0 - \left( p + \rho g l \right) \right) \cdot S_1 + mg.
\end{equation}
Решим систему из уравнений (\ref{eq72-1}) и (\ref{eq72-2}). Избавимся от величины $p-p_0$, так как она содержит неизвестное нам $p$, которое найти не требуется. Выразим эту величину из уравнения (\ref{eq72-1})
\[
	p-p_0 = -\frac{k \cdot l}{S_2}.
\]
Подставим эту величину во второе уравнение и получим
\[
	k \cdot l = \left( \frac{k \cdot l}{S_2} - \rho g l \right) S_1 + mg.
\]
Если выразить отсюда $l$, то получится, что 
\[
	l = \frac{mg}{k \left(1 - \frac{S_1}{S_2}\right) + \rho g S_1}.
\]
Подставим сюда численные значения
\[
	l = \frac{1~\text{кг} \cdot 10~\text{Н/кг}}{-50~\text{Н/м}  + 1000~\text{кг/м}^3 \cdot 10~\text{Н/кг} \cdot 0{,}015~\text{м}^2} = 0{,}1~\text{м}.
\]

\underline{\textbf{Ответ:}} $l = 10$~см. 
