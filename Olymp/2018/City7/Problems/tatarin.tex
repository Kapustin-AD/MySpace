	В одной древней цивилизации существовал обычай, по которому любой человек мог получить себе такой квадратный участок земли, который он сможет обежать за сутки. Один из них рассчитал маршрут так, чтобы добежать ровно за $24$~часа. Однако, через $3$~часа он понял, что переоценил свои силы и оставшееся время сможет бежать только с вдвое меньшей скоростью. Не растерявшись, он быстро перестроил маршрут и добежал вовремя. Найдите отношение площади участка, который он получил, к площади участка, который он хотел получить изначально.  