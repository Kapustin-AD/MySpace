%	Некому члену жюри поручили проверить первую задачу районного этапа олимпиады по физике у всего седьмого класса. Работ очень много, и в каждой пятой работе первая задача не написана. В таком случае проверяющий ставит прочерк в таблице и не тратит времени на проверку. За остальные работы он ставит либо 0 баллов (время на проверку --- 20 c), либо 10 (время на проверку --- 100 с). Он проверил $1000$ работ, затратив в среднем $64$~секунды на работу. Какая доля чернил останется в ручке проверяющего, если она расчитана на 100 м непрерывной чернильной линии? Расход чернил на различные отметки проверяющего указан на рисунке, сторона клетки равна $0{,}5$~см.
	Василий проверяет первую задачу районного этапа олимпиады по физике во всех работах седьмого класса. Оказалось, что в каждой пятой работе первая задача не написана. В таком случае проверяющий ставит прочерк в таблице результатов (см. рис.), не тратя времени на проверку. В остальных работах он ставит либо $0$ баллов (время проверки --- $20$~c), либо $10$ (время проверки --- $100$~с). Проверив $1000$ работ, Василий затратил в среднем $64$~секунды на работу. Какая доля чернил останется в ручке проверяющего, если она рассчитана на $100$~м непрерывной чернильной линии? Василий всегда одинаковым образом ставит прочерки, $0$ и $10$, пример отметок приведен на рисунке, сторона клетки $0{,}5$~см.