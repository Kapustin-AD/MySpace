Миша измерял скорость улитки, ползущей вдоль веревки, привязанной к столбу (см. рис.). Каждые $100$~секунд он отмечал положение улитки, очень быстро завязывая рядом с ней узелок. В какой-то момент ему стало скучно, и он стал завязывать по два узелка в одном месте, вместо одного. Вернувшись домой с веревкой, он измерил расстояния между узелками (не развязывая их) и, поделив их на $100$~секунд, вычислил скорости движения улитки. Если брать расстояние между одинарными узелками, скорость оказалась равна $1{,}3$~мм/с, а если между двойными --- $1{,}1$~мм/с. Найдите скорость движения улитки, если на протяжении всего пути она оставалась постоянной. Все узелки одинаковые. 