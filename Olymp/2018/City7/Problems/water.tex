	Имеется сосуд, заполненный водой и закрытый двумя поршнями площадью $S_1 = 150~\text{см}^2$ и $S_2 = 75~\text{см}^2$, верхний из которых очень легкий, а нижний имеет массу $m = 1$~кг. Поршни связаны пружиной с жесткостью $k = 50~\text{Н/м}$. Найдите удлинение пружины в положении равновесия системы, если в нерастянутом состоянии ее длина пренебрежимо мала. Атмосферное давление $p_0 = 100$~кПа.