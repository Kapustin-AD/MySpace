	Мальчик Карл, прослушав лекцию про черные дыры, запомнил, что радиус черной дыры пропорционален ее массе с некоторой константой $\alpha$, которую он не запомнил. Чтобы лучше овладеть этой темой для школьного доклада, он решил посчитать, какого размера получится черная дыра из воды плотностью $\rho = 1~\text{г/см}^3$. Для этого он нашел в учебнике график зависимости средней плотности черной дыры от ее радиуса, представленный на рисунке и формулу для рассчета объема шара $V = \frac{4}{3}\pi R^3$.