\input{./input/main}

\begin{document}
\begin{center}
{\large Завхоз и мыло\\
}
\end{center}

\subsection*{Условие (V 1.0)}

Завхоз купил в {\itshape столовую} большую бутылку жидкого мыла. Через неделю он обнаружил, что все мыло кончилось. Тогда завхоз купил такую же новую бутылку, но в целях экономии решил доливать в нее воду доверху каждый раз, когда уровень жидкости опускается до середины. Известно, что каждый раз посетители чувствуют, что мыло разбавили в 2 раза, и начинают выдавливать в 2 раза больше порций. Через сколько дней завхозу придется купить новое мыло? Считайте, что посетители в столовую приходят равномерно и изначально все выдавливали одну порцию мыла.

\subsection*{Условие (V 1.0)}

Завхоз купил в {\itshape столовую} большую бутылку жидкого мыла. Через неделю он обнаружил, что все мыло кончилось. Тогда завхоз купил новую бутылку, но в целях экономии решил периодически разбавлять мыло водой, доливая ее в бутылку. Известно, что все посетители чувствуют, во сколько раз разбавлено мыло, и начинают выдавливать во столько же раз больше порций. Через сколько дней завхозу придется купить новое мыло? Считайте, что посетители в столовую приходят равномерно и исходно выдавливали по одной порции мыла каждый.

\subsection*{Решение}

Пусть в какой-то момент концентрация <<чистого>> мыла в бутылке была равна $n$, а жидкость занимала $1/k$-ю часть {\itshape (половину)} бутылки, тогда после разбавления концентрация мыла станет равна  $n/k$ {\itshape ($n/2$)}. При этом если до равзбавления объем мыла, выдавливаемый посетителем, был равен $V$, то после разбавления объем увеличится в $k$ раз, то есть станет равным $k V$ {\itshape ($2 V$)}. Посчитаем, сколько <<чистого>> мыла выдавливал посетитель. Исходно было 

\begin{equation}
V_\text{ч, 1} = V \cdot n
\end{equation}

После разбавления стало

\begin{equation}
V_\text{ч, 2} = k V \cdot n / k = V \cdot n = V_\text{ч, 1}
\end{equation}

Таким образом мы видим, что объем <<чистого>> мыла, выдавливаемого посетителями, от разбавления не изменяется, а значит если раньше оно кончилось за неделю, то и при разбавлении оно кончится тоже за неделю.

\end{document}

