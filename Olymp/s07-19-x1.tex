\documentclass[11pt,a4paper,onecolumn]{article}

\usepackage{amssymb,amsmath,amsfonts}
\usepackage{bm}
\usepackage[margin=1.5cm]{geometry}			
\usepackage{indentfirst,misccorr}
\usepackage{calc}
\pagestyle{empty}

\setlength{\parindent}{0pt}
\setlength{\parskip}{0pt}

% ШРИФТЫ
\usepackage{ifpdf}
% if pdflatex, then
\ifpdf
	\usepackage[utf8]{inputenc}
	\usepackage[T2A]{fontenc}
	\usepackage[russian]{babel}
	\usepackage[unicode]{hyperref}
	\usepackage[pdftex]{graphicx}
	%\usepackage[pdftex]{graphicx}
%	\usepackage{cmlgc}
\else % if xelatex, then
	\usepackage{unicode-math}
	%\usepackage{fouriernc}
	% xelatex specific packages
	\usepackage[xetex]{hyperref}
	\usepackage{xltxtra} % \XeLaTeX macro
	\usepackage{xunicode} % some extra unicode support
	\defaultfontfeatures{Mapping = tex-text}
	\usepackage{polyglossia} % instead of babel in xelatex
	\usepackage{indentfirst} %
	\setdefaultlanguage[spelling = modern]{russian}
	\setotherlanguage{english} %% объявляет второй язык документа
	\newfontfamily\cyrillicfont{Linux Libertine}
	\setmainfont{Linux Libertine}
	\setmonofont{Consolas}
	\setsansfont{Linux Biolinum O}
	\setmathfont{Cambria Math}
\fi


% УСЛОВИЯ ЗАДАЧ
\usepackage{tabularx}
\usepackage{makecell}
\usepackage{enumitem}
% счётчик задач
\newcounter{notask}
\setcounter{notask}{1}

% \task{УСЛОВИЕ ЗАДАЧИ}
% задача без картинки
% оформлена как таблица с двумя колонками
% ширина первой колонки (номер столбца) фиксирована, 0.3cm
% ширина второй колонки автоматически рассчитывается из ширины
% страницы (с учётом всевозможных отступов)
\newcommand{\task}[1]{
	\begin{tabularx}{\textwidth}{|c|X|}
		\cline{1-2}
		\makecell*[{{p{0.5cm}}}]{ \centering \arabic{notask}} &
		\makecell*[{{p{\hsize}}}]{ #1 } \\
		\cline{1-2}
	\end{tabularx}
	
	\vspace{-1pt}
	
	\addtocounter{notask}{1}
}


% \taskpic[ШИРИНА КАРТИНКИ]{УСЛОВИЕ ЗАДАЧИ}{КАРТИНКА}
% задача с картинкой
% оформлена как таблица с тремя колонками
% первый аргумент - необязательный, по умолчанию ширина картинки равна
% 4cm, но можно выставить свою
% ширина второй колонки (условие задачи) рассчитывается из ширины
% страницы и ширины картинки
\newcommand{\taskpic}[3][4cm]{
	\begin{tabularx}{\textwidth}{|c|X|c|}
		\cline{1-3}
		\makecell*[{{p{0.5cm}}}]{ \centering \arabic{notask}} &
		\makecell*[{{p{\hsize}}}]{ #2 } &
		\makecell*[{{p{#1}}}]{ \centering #3} \\
		\cline{1-3}
	\end{tabularx}
	
	\vspace{-1pt}
	
	\addtocounter{notask}{1}
}


% Здесь идут команды для tikz если они будут конфликтными надо расскоментировать следующую строчку
%\endinput
\usepackage{tikz}
\usepackage{pgfplots}
\usepackage{wrapfig}
\usepackage{subfig}

% основные библиотеки tikz
\usetikzlibrary{
	arrows,
	calc,
	patterns,
	intersections,
	decorations.pathreplacing,
	decorations.pathmorphing,
	decorations.text,
	decorations.markings,
	shapes,
	positioning
}

% библиотеки для электричества
\usetikzlibrary{circuits.ee, circuits.ee.IEC}

% для работы с графиками            
\pgfplotsset{compat = newest}

% основные используемые стили
% стиль для стрелки
\tikzset{
	>=latex,
	% платформа: пол или потолок  
	interface/.style = {
		postaction = {
			draw,
			decorate,
			decoration = {
				border,
				angle = 45,
				amplitude = 0.2cm,
				segment length = 1mm
			}
		}
	},
	% пружина  
	spring/.style = {
		decorate,
		decoration = {
			coil,
			amplitude = 1mm,
			segment length = 1mm
		},
		thick
	},
	% заряд, вершина, просто точка
	dot/.style = {
		inner sep = 0mm,
		minimum size = 0.18cm,
		fill,
		circle
	},
	% стрелка в середине отрезка  
	marrow/.style = {
		postaction = {
			draw,
			decorate,
			decoration = {
				markings,
				mark = at position 0.6 with {\arrow{latex}}
			}
		}
	}
}


\tikzstyle{dotnode} = [draw, fill, inner sep = 0pt, minimum size = 1mm, circle]

\tikzstyle{termnode} = [draw, fill = white, inner sep = 0pt, minimum size = 1.5mm, circle]

\newcommand{\termcircuit}[2]{
	\node[termnode] at ({#1}, {#2}) {};
	\draw ({#1 + 0.12}, {#2 + 0.15}) -- ({#1 - 0.12}, {#2 - 0.15});
}




% обозначение угла
\tikzset{
	arcnode/.style = {
		decoration = {
			markings,
			raise = 2mm,
			mark = at position 0.5 with {
				\node[inner sep = 0] {#1};
			}
		},
		postaction = {decorate}
	}
}

% команда для отметки угла: проводит дугу между двумя лучами,
% проведёнными между точками #2--#3 и #2--#4
% первый аргумент - необязательный, стиль линии
% #2,#3,#4 - точки
% #5 - радиус дуги для обозначения угла
% #6 - обозначение угла, например, $\alpha$
\newcommand*\marktheangle[6][]{
	\draw[thick, arcnode = {#6}, #1] let \p2 = ($(#3)-(#2)$),
	\p3 = ($(#4)-(#2)$),
	\n2 = {atan2(\x2,\y2)},
	\n3 = {atan2(\x3,\y3)}%
	in ($(\n2:#5)+(#2)$) arc (\n2:\n3:#5);
}


% амперметр
\tikzset{circuit declare symbol = ammeter}
\tikzset{
	set ammeter graphic = {
		draw,
		generic circle IEC,
		minimum size = 5mm,
		info=center:A
	}
} 

% вольтметр
\tikzset{circuit declare symbol = voltmeter}
\tikzset{
	set voltmeter graphic = {
		draw,
		generic circle IEC,
		minimum size = 5mm,
		info=center:V
	}
} 

% кружок
\tikzset{circuit declare symbol = meter}
\tikzset{
	set meter graphic = {
		draw,
		generic circle IEC,
		minimum size = 5mm
	}
} 


\newcommand{\platform}[2]{
	\draw [draw = none, pattern = north west lines] 
	#1 -- ( $#1!0.15cm!90:#2$ ) -- ( $#2!0.15cm!-90:#1$ ) -- #2;
	\draw #1 -- #2
}


%\input{preambula_usep4ackage}
\setlength{\parindent}{12pt}
\setlength{\parskip}{0pt}
\begin{document}

\section*{Вариант 1}

\subsection*{Условие}

Автомобиль выехал из населенного пункта \textit{A} в пункт \textit{D} по дороге проезжая пункты \textit{B} и \textit{С}. Расстояние между всеми соседними пунктами 60 км (см. Рис). Максимально разрешенная скорость на дороге между пунктами \textit{A} и \textit{B} составляет 60 км/ч, между \textit{В} и \textit{С} 90 км/ч, между \textit{C} и \textit{D} 120 км/ч. Для контроля за скоростью движения вдоль дороги поставлены три камеры, которые фиксируют в какое время мимо нее проезжает автомобиль: первая камера в пункте A, вторая посередине между B и C, и третья в пункте D. Автомобиль проезжает первую в 13.20, вторую в 14.30 и третью в 15.30. Можно ли сказать, что автомобиль наверняка превышал максимальную скорость, и, если да, то на каком участке? Ответ обоснуйте.

\subsection*{Решение}
     Можно посчитать, за какое минимальное время может проехать автомобиль от первой до второй камеры, не нарушая правил дорожного движения. Для этого автомобиль должен двигаться с максимально разрешенной скоростью, то есть 60 км/ч на отрезке AB на протяжении 60 км и со скоростью 90 км/ч на половине отрезка ВС на протяжении еще 30 км. Сумма времен на прохождение расстояния между первой и второй камерами будет равна $T_1$:
     \begin{equation}
     T_1 = \frac{60 \mbox{км}}{60 \mbox{км/ч}} + \frac{30 \mbox{км}}{90 \mbox{км/ч}} = 1 \mbox{ час } 20 \mbox{ минут } \mbox{ . }
     \end{equation}
Так как автомобиль проехал этот отрезок за 1 час 10 минут, очевидно, что он нарушил правила на отрезке между первой и второй камерой. Исходя из аналогичных соображений, можно рассчитать минимальное время движения между второй и третьей камерами $T_2$:
     \begin{equation}
     T_2 = \frac{30 \mbox{км}}{90 \mbox{км/ч}} + \frac{60 \mbox{км}}{120 \mbox{км/ч}} = 50 \mbox{ минут } \mbox{ . }
     \end{equation}
Так как автомобиль ехал отрезок между второй и третьей камерой 60 минут, нельзя сказать, что он наверняка превышал скорость на этом отрезке.

\textit{Ответ:} Автомобиль превышал скорость на отрезке между первой и второй камерами.


\section*{Вариант 2}

\subsection*{Условие}

     Автомобиль выехал из населенного пункта \textit{A} в пункт \textit{D} по дороге проезжая пункты \textit{B} и \textit{С}. Расстояние между всеми соседними пунктами 100 км (см. Рис). Максимально разрешенная скорость на дороге между пунктами \textit{A} и \textit{B} составляет 50 км/ч, между \textit{В} и \textit{С} 75 км/ч, между \textit{C} и \textit{D} 100 км/ч. Для контроля за скоростью движения вдоль дороги поставлены три камеры, которые фиксируют в какое время мимо нее проезжает автомобиль: первая камера в пункте A, вторая посередине между B и C, и третья в пункте D. Автомобиль проезжает первую в 16.30, вторую в 19.30 и третью в 21.00. Можно ли наверняка сказать, что автомобиль превышал максимальную скорость, и, если да, то на каком участке? Ответ обоснуйте.
     
     
\subsection*{Решение}
Можно посчитать, за какое минимальное время может проехать автомобиль от первой до второй камеры, не нарушая правил дорожного движения. Для этого автомобиль должен двигаться с максимально разрешенной скоростью, то есть 50 км/ч на отрезке AB на протяжении 100 км и со скоростью 75 км/ч на половине отрезка ВС на протяжении еще 50 км. Сумма времен на прохождение расстояния между первой и второй камерами будет равна $T_1$:
     \begin{equation}
     T_1 = \frac{100 \mbox{км}}{50 \mbox{км/ч}} + \frac{50 \mbox{км}}{75 \mbox{км/ч}} = 2 \mbox{ часа } 40 \mbox{ минут } \mbox{ . }
     \end{equation}
Так как автомобиль проехал этот отрезок за 3 часа нельзя сказать, что он наверняка превышал допустимую скорость. Исходя из аналогичных соображений, можно рассчитать минимальное время движения между второй и третьей камерами $T_2$:
     \begin{equation}
     T_2 = \frac{50 \mbox{км}}{75 \mbox{км/ч}} + \frac{100 \mbox{км}}{100 \mbox{км/ч}} = 1 \mbox{ час } 40 \mbox{ минут } \mbox{ . }
     \end{equation}
Так как автомобиль ехал отрезок между второй и третьей камерой 1 час 30 минут, то он точно превышал скорость на этом отрезке.

\textit{Ответ:} Автомобиль превышал скорость на отрезке между второй и третьей камерами.



\end{document}
