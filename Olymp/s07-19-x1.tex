\input{preambula}
%\input{preambula_usep4ackage}
\setlength{\parindent}{12pt}
\setlength{\parskip}{0pt}
\begin{document}

\section*{Вариант 1}

\subsection*{Условие}

Автомобиль выехал из населенного пункта \textit{A} в пункт \textit{D} по дороге проезжая пункты \textit{B} и \textit{С}. Расстояние между всеми соседними пунктами 60 км (см. Рис). Максимально разрешенная скорость на дороге между пунктами \textit{A} и \textit{B} составляет 60 км/ч, между \textit{В} и \textit{С} 90 км/ч, между \textit{C} и \textit{D} 120 км/ч. Для контроля за скоростью движения вдоль дороги поставлены три камеры, которые фиксируют в какое время мимо нее проезжает автомобиль: первая камера в пункте A, вторая посередине между B и C, и третья в пункте D. Автомобиль проезжает первую в 13.20, вторую в 14.30 и третью в 15.30. Можно ли сказать, что автомобиль наверняка превышал максимальную скорость, и, если да, то на каком участке? Ответ обоснуйте.

\subsection*{Решение}
     Можно посчитать, за какое минимальное время может проехать автомобиль от первой до второй камеры, не нарушая правил дорожного движения. Для этого автомобиль должен двигаться с максимально разрешенной скоростью, то есть 60 км/ч на отрезке AB на протяжении 60 км и со скоростью 90 км/ч на половине отрезка ВС на протяжении еще 30 км. Сумма времен на прохождение расстояния между первой и второй камерами будет равна $T_1$:
     \begin{equation}
     T_1 = \frac{60 \mbox{км}}{60 \mbox{км/ч}} + \frac{30 \mbox{км}}{90 \mbox{км/ч}} = 1 \mbox{ час } 20 \mbox{ минут } \mbox{ . }
     \end{equation}
Так как автомобиль проехал этот отрезок за 1 час 10 минут, очевидно, что он нарушил правила на отрезке между первой и второй камерой. Исходя из аналогичных соображений, можно рассчитать минимальное время движения между второй и третьей камерами $T_2$:
     \begin{equation}
     T_2 = \frac{30 \mbox{км}}{90 \mbox{км/ч}} + \frac{60 \mbox{км}}{120 \mbox{км/ч}} = 50 \mbox{ минут } \mbox{ . }
     \end{equation}
Так как автомобиль ехал отрезок между второй и третьей камерой 60 минут, нельзя сказать, что он наверняка превышал скорость на этом отрезке.

\textit{Ответ:} Автомобиль превышал скорость на отрезке между первой и второй камерами.


\section*{Вариант 2}

\subsection*{Условие}

     Автомобиль выехал из населенного пункта \textit{A} в пункт \textit{D} по дороге проезжая пункты \textit{B} и \textit{С}. Расстояние между всеми соседними пунктами 100 км (см. Рис). Максимально разрешенная скорость на дороге между пунктами \textit{A} и \textit{B} составляет 50 км/ч, между \textit{В} и \textit{С} 75 км/ч, между \textit{C} и \textit{D} 100 км/ч. Для контроля за скоростью движения вдоль дороги поставлены три камеры, которые фиксируют в какое время мимо нее проезжает автомобиль: первая камера в пункте A, вторая посередине между B и C, и третья в пункте D. Автомобиль проезжает первую в 16.30, вторую в 19.30 и третью в 21.00. Можно ли наверняка сказать, что автомобиль превышал максимальную скорость, и, если да, то на каком участке? Ответ обоснуйте.
     
     
\subsection*{Решение}
Можно посчитать, за какое минимальное время может проехать автомобиль от первой до второй камеры, не нарушая правил дорожного движения. Для этого автомобиль должен двигаться с максимально разрешенной скоростью, то есть 50 км/ч на отрезке AB на протяжении 100 км и со скоростью 75 км/ч на половине отрезка ВС на протяжении еще 50 км. Сумма времен на прохождение расстояния между первой и второй камерами будет равна $T_1$:
     \begin{equation}
     T_1 = \frac{100 \mbox{км}}{50 \mbox{км/ч}} + \frac{50 \mbox{км}}{75 \mbox{км/ч}} = 2 \mbox{ часа } 40 \mbox{ минут } \mbox{ . }
     \end{equation}
Так как автомобиль проехал этот отрезок за 3 часа нельзя сказать, что он наверняка превышал допустимую скорость. Исходя из аналогичных соображений, можно рассчитать минимальное время движения между второй и третьей камерами $T_2$:
     \begin{equation}
     T_2 = \frac{50 \mbox{км}}{75 \mbox{км/ч}} + \frac{100 \mbox{км}}{100 \mbox{км/ч}} = 1 \mbox{ час } 40 \mbox{ минут } \mbox{ . }
     \end{equation}
Так как автомобиль ехал отрезок между второй и третьей камерой 1 час 30 минут, то он точно превышал скорость на этом отрезке.

\textit{Ответ:} Автомобиль превышал скорость на отрезке между второй и третьей камерами.



\end{document}
