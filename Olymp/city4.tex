\par Поймём, как двигается тень. Для этого закрепим фонарь в некоторой точке столба, но выше верхней точки траектории снежка, иначе тень уйдёт на бесконечность. Будем отмечать положения тени при различных положенииях снежка. Заметим, что сначала тень двигается от столба, а затем к нему. Точке разворота соответствует момент, когда снежок летит прямо на фонарь, т.е. когда лучи света идут по касательной к траектории снежка.
\par Пусть $H$~--- высота столба, $h$~--- максимальная высота подъёма снежка, $L$~--- дальность полета снежка, а $l$~--- путь тени при её движении от столба. Путь тени тогда складывается из двух частей:
\begin{equation}\label{eq:8-s=L+2l}
	s = L + 2 l . 
\end{equation}
\par По условию:
\begin{equation}
	\frac{L}{s} = \frac{h}{H} = \frac{2{,}5~\textup{м}}{5~\textup{м}} = \frac{1}{2}.
\end{equation}
\par Подставляем (\ref{eq:8-s=L+2l}) и находим $l$:
\begin{equation}
	\frac{L}{L+2l}=\frac{1}{2}
	\;\;\Rightarrow\;\;
	2L = L + 2l
	\;\;\Rightarrow\;\;
	l = \frac{L}{2} = \frac{5~\text{м}}{2} = 2{,}5~\text{м}.
\end{equation}
\par Отмечаем точку разворота тени на расстоянии $l$ справа от точки броска и проводим через неё касательную к траектории снежка. Касательная пересечёт столб в точке крепления фонаря. Получим, что высота крепления фонаря равна $4~\textup{м}$.
\Large добавить слов\normalsize
\begin{figure}[h]\centering
\begin{tikzpicture}
	\draw [help lines, step = .5] (-1.5,0) grid (8, 5);
	\platform{(7.5,0)}{(-.5,0)};
	\draw [scale=1,domain=0:5,smooth,variable=\x, very thick] plot 
		({\x}, {-(\x-5) * (\x)/2.5});
	\draw [<->, thick] (-.2,0) to node [midway, left, fill = white] {$5$~м} ++ (0, 5);
	\draw [line width = 3pt] (0,0) to (0,5);
	\draw (6, 3) node [fill = white] (L) {траектория снежка};
	\draw [->] (L.south) to [out = -90, in = 10] (4.4,1.2);
	\draw [very thick, black] (7.5,0) to (0,4);
	\draw [blue, ->, thick] (5, - .2) to ++ (2, 0);
	\draw [blue, ->, thick] (7.1, -.2) .. controls (7.9, -.3) and 
							(7.9, -.4) .. (5,-.4);
	\draw [blue, ->, thick] (5,-.4) to ++ (-5, 0);
	\draw [blue] (3.5, -.4) node [below] {путь тени};
	\draw [blue, <->] (0,.2) to node [midway, above,fill = white] {$L$} ++ (5,0);
	\draw [blue, <->] (5,.2) to node [pos=.3, above,fill = white] {$l$} ++ (2.5,0);
\end{tikzpicture}
\caption{Построение положения лампы.}
\end{figure}

\olympanswer{ фонарь находится на высоте $4$~м от земли.}

\ifgrade
\begin{grade-env}
	\grade{2}{Описан характер движения тени. Сначала вправо, потом влево.}
	\grade{1}{Найдена точка, с максимальным удалением от столба.}
	\grade{1}{Построено положение лампы на фонаре.}
\end{grade-env}
\fi