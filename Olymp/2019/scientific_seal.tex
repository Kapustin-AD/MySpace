Для исследования свойств морской воды ученые закрепили на спине тюленя всевозможные датчики и отпустили его на свободу в море. На рисунке приведены показания датчика давления в зависимости от времени. Определите перемещение тюленя за время наблюдения. Известно, что он не поворачивал направо или налево. Во всех направлениях его скорость постоянна и равна  $1{,}5$ м/с, плотность морской воды 1000~кг/м$^3$, ускорение свободного падения --- 10~м/c, атмосферное давление --- 100~кПа.

В ходе исследований ни один тюлень не пострадал.