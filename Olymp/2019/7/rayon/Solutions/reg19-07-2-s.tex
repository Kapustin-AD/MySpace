\subsubsection{Первое возможное решение}

Изначально в бутылке было только мыло. Тогда каждый посетитель, выдавливая объем $V$, тратил ровно такой же объем мыла. После первого разбавления в бутылке оказалась смесь мыла и воды, в которой однут \variants{треть}{четверть} занимает мыло и \variants{две трети}{три четверти} --- вода. Тогда каждый посетитель, выдавливая тот же объем $V$, тратил бы лишь $\variants{V/3}{V/4}$ мыла. Однако, каждый посетитель после разбавления стал выдавливать объем $\variants{3V}{4V}$ жидкости, и поэтому тратить тот же самый объем мыла $V$, что и до разбавления. Поэтому расход мыла не изменился после первого разбавления. 

Сразу после второго разбавления в бутылке находится только $\variants{1/9}{1/16}$ часть мыла и $\variants{8/9}{15/16}$ частей воды. Тогда выдавливаемое количество мылы составляет $\variants{1/9}{1/16}$ часть от объема выдавленной жидкости. Однако теперь посетители выдавливают еще в \variants{три}{четыре} раза больше жидкости, то есть в $\variants{9}{16}$ раз больше, чем изначально. Это значит, что они снова каждый раз выдавливают такое же количество мыла. Значит расход мыла не меняется и после второго разбавления. 

Можно понять, что расход мыла не изменяется при любом количестве разбавлений. А тогда мыло в банке кончится за то же время, что и в случае, когда его не разбавляли водой, то есть за неделю. 

\olympanswer{через 7 дней.}

\ifgrade
\begin{grade-env}
	\grade{4}{После первого разбавления расход мыла не меняется.}
	\grade{6}{Утверждение, что расход мыла одинаков после любого количества разбавлений.}
\end{grade-env}
\fi

\subsubsection{Второе возможное решение}

Введем числовую ось и обозначим на ней начальный момент, и момент $T$, когда мыло кончилось в первой бутылке. Найдем на ней время $t_1$, в которое произошло первое разбавление. Это произошло, когда израсходовалось $\variants{2/3}{3/4}$ бутылки, то есть прошло $\variants{2T/3}{3T/4}$. 
\begin{figure}[h!]
	\centering
	\begin{tikzpicture}
		\draw[very thick] (0, 0)node[below]{$0$} -- (9, 0)node[below]{$T$};
		\fill (0, 0) circle (0.05);
		\fill (9, 0) circle (0.05);
		\fill[red] (\variants{9-9/3}{9-9/4}, 0)node[above]{$t_1$} circle (0.05);
	\end{tikzpicture}
\end{figure}

Теперь в момент $t_1$ у нас снова имеется полная бутылка жидкости, но расход жидкости увеличился втрое. Тогда до следующего разбавления пройдет \variants{втрое}{вчетверо} меньший промежуток времени. Но можно заметить, что оставшееся время до момента $T$ тоже \variants{втрое}{вчетверо} меньше изначального. Тогда можно сказать, что до следующего разбавления пройдет $\variants{2/3}{3/4}$ оставшегося времени. По этому правилу можно найти момент $t_2$ второго разбавления и моменты $t_n$ всех последующих.
\begin{figure}[h!]
	\centering
	\begin{tikzpicture}
		\draw[very thick] (0, 0)node[below]{$0$} -- (9, 0)node[below]{$T$};
		\fill (0, 0) circle (0.05);
		\fill (9, 0) circle (0.05);
		\fill[red] (\variants{9-9/3}{9-9/4}, 0)node[above]{$t_1$} circle (0.05);
		\fill[blue] (\variants{9-9/9}{9-9/16}, 0)node[above]{$t_2$} circle (0.05);
		\fill[blue] (\variants{9-9/27}{9 - 9/64}, 0)node[above]{$t_3$} circle (0.05);
		\fill[blue] (\variants{9-9/81}{9 - 9/256}, 0) circle (0.05);
	\end{tikzpicture}
\end{figure}

Тогда можно заметить, что каждое разбавление будет происходить все ближе к моменту $T$, но не позже. Кроме того, с каждым разбавлением мыла в бутылке становится все меньше, а значит в момент $T$ его не останется вовсе.
\olympanswer{через 7 дней.}

\ifgrade
\begin{grade-env}
	\grade{4}{Построение первого момета времени.}
	\grade{6}{Правило построения последующих моментов времени.}
\end{grade-env}
\fi
