	Эскалатор можно рассмотреть как два параллельных пути, по которым люди спускаются либо стоя, либо идя пешком. Понятно, что спускаться пешком быстрее. Пусть стоя спускается $N$~человек в минуту, а пешком --- $M~$~человек в минуту. Тогда максимум может спускаться $2M$, минимум $2N$, а в стационарном режиме спускается $N+M$ человек в минуту. 
	\begin{figure}[h!]
	\centering
	\variants{
	\begin{tikzpicture}[scale=0.92,
	/pgfplots/axis labels at tip/.style={ 
		xlabel style={ at={ (current axis.right of origin) }, 
			yshift = 1 ex,
			anchor = south east,
			fill=white}, 
		ylabel style={ at={ (current axis.above origin) }, 
			xshift = 1 ex, 
			anchor = north west,
			fill = white} } ]
	\begin{axis}[
	grid style={line width=0.1pt, draw=gray!80},
		line width = 2pt, 
		axis x line = middle,
		axis y line = middle,
		axis labels at tip,
		xmin = 0, xmax = 5.5,
		ymin = 0, ymax = 55,
		xtick = {0, 1, ..., 5},
		ytick = {0, 10, ..., 50},
		xlabel = {$t,\text{ мин }$},
		ylabel = {$L,\text{ чел}$},
		grid = both,
		major grid style={line width = 1.3pt,draw=black!50},
		minor grid style={line width=.4pt, draw=black!50},
		major tick length = 7pt,
	    every major tick/.style={
			black,
	        line width = 2pt,
        },
		minor tick length = 4pt,
	    every minor tick/.style={
			black,
	        line width = 1pt,
        },
		minor tick num = 0]
	\draw [line width = 2pt] 
		(0, 0) to (1, 10) to (2, 50);
	\draw [line width = 2pt, red] (2, 50) to (4.5, 0);		
	\end{axis}
	\end{tikzpicture}
	}
	{
	\begin{tikzpicture}[scale=0.92,
	/pgfplots/axis labels at tip/.style={ 
		xlabel style={ at={ (current axis.right of origin) }, 
			yshift = 1 ex,
			anchor = south east,
			fill=white}, 
		ylabel style={ at={ (current axis.above origin) }, 
			xshift = 1 ex, 
			anchor = north west,
			fill = white} } ]
	\begin{axis}[
	grid style={line width=.1pt, draw=gray!80},
		line width = 2pt, 
		axis x line = middle,
		axis y line = middle,
		axis labels at tip,
		xmin = 0, xmax = 19,
		ymin = 0, ymax = 150,
		xtick = {0, 2, ..., 18},
		ytick = {0, 20, ..., 140},
		xlabel = {$t,\text{ мин }$},
		ylabel = {$L,\text{ чел}$},
		grid = both,
		major grid style={line width = 1.3pt,draw=black!50},
		minor grid style={line width=.4pt, draw=black!50},
		major tick length = 7pt,
	    every major tick/.style={
			black,
	        line width = 2pt,
        },
		minor tick length = 4pt,
	    every minor tick/.style={
			black,
	        line width = 1pt,
        },
		minor tick num = 0]
	\draw [line width = 2pt] 
		(0, 0) to (2, 40) to (4, 140);
	\draw [line width = 2pt, red] (4, 140) to (18, 0);	
	\end{axis}
	\end{tikzpicture}
	}
\end{figure}

Скорость роста очереди на каждом участкке можно определить по графику, поделив изменение ее длины на время, за которое оно произошло. Так, скорость роста очереди на первом участке равна $\variants{10}{20}$ человек в минуту, на втором --- $\variants{40}{50}$ человек в минуту. 

Эту скорость можно связать с количеством спускающихся людей и числом входящих на станцию. Если за минуту на станцию входит $K$ человек, а спускается изначально $N+M$, то скорость роста очереди на первом участке будет равна $K-(N+M)$. 

На втором участке люди стоят на обеих сторонах, а значит всего спускается $2N$ человек в минуту. Скорость роста очереди на втором участке равна $K - 2N$ и на $30$ человек в минуту больше, чем на первом. Значит $M-N = 30$ человек в минуту. 

После двух минут люди стали идти пешком по двум сторонам, и всего спускется $2M$ человек в минуту. Скорость роста очереди на этом участке равна $K - 2M$. Это на $30$ человек в минуту меньше, чем на первом участке, а значит равна $\variants{-20}{-10}$ человек в минуту. 

По графику заметим, что через $\variants{2}{4}$ минуты у эскалатора образовалась очередь из $\variants{50}{140}$ человек. Тогда со скоростью $\variants{20}{10}$ человек в минуту эта очередь сократится до нуля за 
\begin{equation}
	t = \frac{\variants{50}{140}~\text{чел}}{\variants{20}{10}~\text{чел/мин}} = \variants{2{,}5}{14}~\text{мин}.
\end{equation}

\olympanswer{ за \variants{$2{,}5$ минуты}{$14$ минут}.}

\ifgrade
\begin{grade-env}
	\grade{3}{Умение находить скорость по графику.}
	\grade{5}{Нахождение скорости сокращения очереди в третьем режиме.}
	\grade{2}{Правильный ответ.}
\end{grade-env}
\fi