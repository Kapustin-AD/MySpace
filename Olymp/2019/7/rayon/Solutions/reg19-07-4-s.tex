	Из того, что расстояние между всеми дырками одинаковое, можно сделать вывод о том, что за промежуток времени между выстрелами точка попадания сдвигается на постоянную величину $x$. То есть, при стрельбе по неподвижному поезду расстояние между любыми соседними дырками было бы равно $x$. 
	
	В случае, когда поезд движется \variants{налево}{направо}, расстояние между дырками складывается из $x$ и расстояния $y$, которое проходит поезд
\begin{equation}
	x + y = \variants{100}{140}~\text{см}.
\end{equation}
	
	Если же поезд движется направо, то возможны два варианта. Поезд может проходить расстояние $y > x$, либо $y < x$. В обоих случаях расстояние между дырками будет равно $\variants{30}{100}~\text{см}$, и будет равно разности $x$ и $y$.
	
	В первом сучае
\begin{equation}
	x - y = \variants{30}{100}~\text{см},
\end{equation}
	во втором
\begin{equation}
	y - x = \variants{30}{100}~\text{см}.
\end{equation}
	
	В обоих случаях можно найти интересующий нас $y$. В первом случае $y = \variants{35}{20}~\text{см}$, во втором --- $\variants{65}{120}~\text{см}$. Такое перемещение совершает поезд за время между выстрелами $t$, которое равно $1/5$ секунды. Тогда скорость поезда равна
\begin{equation}
	v = \frac{y}{t},
\end{equation}
и составляет $\variants{1{,}75}{1}~\text{м/с}$ для первого случая и $\variants{3{,}25}{6}~\text{м/с}$ для второго.
\olympanswer{$\variants{1{,}75}{1}~\text{м/с}$ и $\variants{3{,}25}{6}~\text{м/с}$.}

\ifgrade
\begin{grade-env}
	\grade{4}{Правильно рассмотрено относительное движение.}
	\grade{2}{Связь расстояния между дырками со скоростью.}
	\grade{2}{Обнаружено наличие двух случаев.}
	\grade{1}{Правильный ответ для первого случая.}
	\grade{1}{Правильный ответ для второго случая.}
\end{grade-env}
\fi
