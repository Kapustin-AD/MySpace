\begin{figure}[h!]
	\centering
	\variants{
	\begin{tikzpicture}[scale=1.5]
		\draw[very thick] (0, 0) -- (3, 0);
		\fill (0, 0)node[above]{$A$} circle (0.05);
		\fill (1, 0)node[above]{$B$} circle (0.05);
		\fill (2, 0)node[above]{$C$} circle (0.05);
		\fill (3, 0)node[above]{$D$} circle (0.05);
		\draw [dashed] (0, 0) -- (0, -1);
		\draw [dashed] (1, 0) -- (1, -1);
		\draw [dashed] (2, 0) -- (2, -1);
		\draw [dashed] (3, 0) -- (3, -1);
		\draw [<->] (0, -1) -- (1, -1)node[midway, below]{\scriptsize{$60\text{~км}$}};
		\draw [<->] (1, -1) -- (2, -1)node[midway, below]{\scriptsize{$60\text{~км}$}};
		\draw [<->] (2, -1) -- (3, -1)node[midway, below]{\scriptsize{$60\text{~км}$}};
		\draw[rounded corners=0.7] (-0.2, 1) rectangle +(0.4, -0.3);
		\draw (-0.05, 0.85) circle (0.1); 
		\draw (0.14, 0.95) ellipse (0.05 and 0.03);
		\draw[rounded corners=0.7] (1.3, 1) rectangle +(0.4, -0.3);
		\draw (1.45, 0.85) circle (0.1); 
		\draw (1.64, 0.95) ellipse (0.05 and 0.03);
		\draw[rounded corners=0.7] (2.8, 1) rectangle +(0.4, -0.3);
		\draw (2.95, 0.85) circle (0.1); 
		\draw (3.14, 0.95) ellipse (0.05 and 0.03);
	\end{tikzpicture}
	}
	{
	\begin{tikzpicture}[scale=1.5]
		\draw[very thick] (0, 0) -- (3, 0);
		\fill (0, 0)node[above]{$A$} circle (0.05);
		\fill (1, 0)node[above]{$B$} circle (0.05);
		\fill (2, 0)node[above]{$C$} circle (0.05);
		\fill (3, 0)node[above]{$D$} circle (0.05);
		\draw [dashed] (0, 0) -- (0, -1);
		\draw [dashed] (1, 0) -- (1, -1);
		\draw [dashed] (2, 0) -- (2, -1);
		\draw [dashed] (3, 0) -- (3, -1);
		\draw [<->] (0, -1) -- (1, -1)node[midway, below]{\scriptsize{$100\text{~км}$}};
		\draw [<->] (1, -1) -- (2, -1)node[midway, below]{\scriptsize{$100\text{~км}$}};
		\draw [<->] (2, -1) -- (3, -1)node[midway, below]{\scriptsize{$100\text{~км}$}};
		\draw[rounded corners=0.7] (-0.2, 1) rectangle +(0.4, -0.3);
		\draw (-0.05, 0.85) circle (0.1); 
		\draw (0.14, 0.95) ellipse (0.05 and 0.03);
		\draw[rounded corners=0.7] (1.3, 1) rectangle +(0.4, -0.3);
		\draw (1.45, 0.85) circle (0.1); 
		\draw (1.64, 0.95) ellipse (0.05 and 0.03);
		\draw[rounded corners=0.7] (2.8, 1) rectangle +(0.4, -0.3);
		\draw (2.95, 0.85) circle (0.1); 
		\draw (3.14, 0.95) ellipse (0.05 and 0.03);
	\end{tikzpicture}
	}
\end{figure}
     
     Можно посчитать, за какое минимальное время может проехать автомобиль от первой до второй камеры, не нарушая правил дорожного движения. Для этого автомобиль должен двигаться с максимально разрешенной скоростью, то есть $\variants{60}{50}\text{~км/ч}$ на отрезке $AB$ на протяжении $\variants{60}{100}\text{~км}$ и со скоростью $\variants{90}{75}\text{~км/ч}$ на половине отрезка $ВС$ на протяжении еще $\variants{30}{50}\text{~км}$. Сумма времен на прохождение расстояния между первой и второй камерами будет равна $T_1$:
     \begin{equation}
     T_1 = \frac{\variants{60}{100}\text{~км}}{\variants{60}{50}\text{~км/ч}} + \frac{\variants{30}{50}\text{~км}}{\variants{90}{75}\text{~км/ч}} = \variants{1\text{~час}~20\text{~минут}}{2\text{~часа}~40\text{~минут}}.
     \end{equation}
Так как автомобиль проехал этот отрезок за \variants{$1$ час $10$ минут}{$3$ часа}, \variants{можно}{нельзя} сказать, что он наверняка нарушил правила на отрезке между первой и второй камерой. 

Теперь рассчитаем минимальное время движения между второй и третьей камерами. Автомобиль может оставшиеся $\variants{30}{50}\text{~км}$ до точки $C$ проехать со скоростью $\variants{90}{75}\text{~км/ч}$, а затем $\variants{60}{100}\text{~км}$ до точки $D$ проехать со скоростью $\variants{120}{100}\text{~км/ч}$. Тогда затраченное на это время будет равно $T_2$:
     \begin{equation}
     T_2 = \frac{\variants{30}{50}\text{~км}}{\variants{90}{75}\text{~км/ч}} + \frac{\variants{60}{100}\text{~км}}{\variants{120}{100}\text{~км/ч}} = \variants{50\text{~минут}}{1\text{~час}~40\text{~минут}}.
     \end{equation}
Так как автомобиль проехал отрезок между второй и третьей камерой за \variants{$60$ минут}{1\text{~час}~30\text{~минут}}, \variants{нельзя}{можно} сказать, что он наверняка превышал скорость на этом отрезке.

\olympanswer{Автомобиль наверняка превышал скорость на отрезке между \variants{первой}{второй} и \variants{второй}{третьей} камерами.}

\ifgrade
\begin{grade-env}
	\grade{3}{Проведены расчеты величины, позволяющей определить, превышал ли автомобиль скорость на первом участке.}
	\grade{2}{\variants{Вывод, что автомобиль \textbf{наверняка} нарушал скорость на первом участке.}{Вывод, что нельзя сказать, что автомобиль \textbf{наверняка} нарушал скорость на первом участке.}}
	\grade{3}{Проведены расчеты величины, позволяющей определить, превышал ли автомобиль скорость на втором участке.}
	\grade{2}{\variants{Вывод, что нельзя сказать, что автомобиль \textbf{наверняка} нарушал скорость на втором участке.}{Вывод, что автомобиль \textbf{наверняка} нарушал скорость на втором участке.}}
\end{grade-env}
\fi
