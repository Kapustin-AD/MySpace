Автомобиль выехал из пункта $A$ в пункт $D$ по дороге проезжая пункты $B$ и $C$. Расстояние между всеми соседними пунктами $100\unit{км}$. Максимально разрешенная скорость на дороге между пунктами $A$ и $B$ составляет $50\unit{км/ч}$, между $B$ и $C$ --- $75\unit{км/ч}$, между $C$ и $D$ --- $100\unit{км/ч}$. Вдоль дороги поставлены три камеры, которые фиксируют, в какое время мимо нее проезжает автомобиль. Первая камера в пункте $A$, вторая посередине между $B$ и $C$, и третья в пункте $D$. Автомобиль проезжает первую в $16$:$30$, вторую в $19$:$30$ и третью в $21$:$00$.  На участках между какими камерами автомобиль точно превысил скоростной режим?