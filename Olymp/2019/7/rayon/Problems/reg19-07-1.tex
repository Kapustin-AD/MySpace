Автомобиль выехал из пункта $A$ в пункт $D$, по дороге проезжая пункты $B$ и $C$. Расстояние между всеми соседними пунктами $60\unit{км}$. Максимально разрешенная скорость на дороге между пунктами $A$ и $B$ составляет $60\unit{км/ч}$, между $B$ и $C$ --- $90\unit{км/ч}$, между $C$ и $D$ --- $120\unit{км/ч}$. Вдоль дороги поставлены три камеры, которые фиксируют, в какое время мимо нее проезжает автомобиль. Первая камера в пункте $A$, вторая посередине между $B$ и $C$, и третья в пункте $D$. Автомобиль проезжает первую в $13$:$20$, вторую в $14$:$30$ и третью в $15$:$30$. На участках между какими камерами автомобиль точно превысил скоростной режим?