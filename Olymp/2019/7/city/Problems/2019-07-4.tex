Система кондиционирования подает в салон пассажирского самолета $U$ = 15 м$^3$ чистого воздуха (без пыли) каждую минуту. С каждого из пассажиров в воздух попадает постоянное число частицы пыли в минуту. Кроме того, с грязного пола в воздух также переходит постоянное число пылинок в минуту. Бортпроводник измеряет концентрацию пыли $C$: количество пылинок в одном кубическом сантиметре воздуха (штук$/$см$^3$). Если удвоить количество подаваемого воздуха $U$, то концентрация пылинок в воздухе уменьшится на $\Delta C_1$ = 100 штук$/$см$^3$. А если удвоить количество пассажиров, то концентрация пылинок в воздухе увеличится на $\Delta C_2$ = 100 штук$/$см$^3$. Сколько частиц пыли в секунду попадает в воздух c грязного пола? \\
Количество воздуха в салоне и концентрация пыли в процессе полета неизменны. Пыль распределяется равномерно по всему объему салона, свойства пола не зависят от количества пассажиров.