Из трубы химического завода города Черноснежинска было выпущено два облака, состоящих из мелких капель двух различных веществ, которые при контакте исчезают, образуя снежинки. Облака имеют форму одинаковых прямоугольных треугольников с длиной основания $100\unit{м}$. Начальное расстояние между облаками $1\unit{км}$, они движутся со скоростями $15\unit{м/с}$ и $5\unit{м/с}$, оставаясь на одной высоте. Найдите продолжительность снегопада и момент времени, когда количество выпадающего снега было максимальным. \\
Реакция капель происходит быстро, разные облака состоят из разных веществ, и концентрация капель в облаках одинакова.