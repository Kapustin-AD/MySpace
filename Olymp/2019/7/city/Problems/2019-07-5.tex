Имеется конструкция из двух кубиков с длиной ребер $10$~см, соединенных пружиной жесткостью $100$~Н/м. Длина нерастянутой пружины $10$~см. Конструкцию помещают в сосуд, в который налито некоторое количество воды плотностью $1000\text{ кг/м}^3$ и масла плотностью $900\text{ кг/м}^3$. Оказывается, что она плавает так, что каждый куб наполовину погружен в масло. Найдите разницу высот между уровнем масла и воды в таком состоянии, если известно, что верхний кубик в три раза легче нижнего.