Для исследования свойств морской воды учёные закрепили на спине дельфина всевозможные датчики и отпустили его на свободу в море. На рисунке приведены показания датчика давления в зависимости от времени. Определите перемещение дельфина за время наблюдения. Известно, что он не поворачивал направо или налево. Скорость его движения была постоянна по величине и равна  $2$ м/с, плотность морской воды $1000$~кг/м$^3$, ускорение свободного падения~--- $10$~м/c$^2$, атмосферное давление~--- $100$~кПа.\\
	В ходе исследований ни один дельфин не пострадал.