	Первым делом заметим, что давление в конце наблюдения равно атмосферному, значит дельфин вынырнул на поверхность и перемещение по вертикали равно нулю. Таким образом необходимо следить только за горизонтальным перемещением.
	 
	Обозначим атмосферное давение $p_A$.
	Тогда давление жидкости на глубине $h$ равно $\rho g h + p_A$, поэтому изменение давления за единицу времени с точностью до $\rho g$ равно вертикальной скорости тюленя
\begin{equation}
	\Delta p = \rho g \Delta h = \rho g v \Delta t, \ \ \ \ \text{или} \ \ \ \ v = \frac{\Delta p}{\rho g \Delta t}.
\end{equation} 
Посчитаем из коэффициента наклона вертикальную составляющую в зависимости от времени.
	
	Можно вычислить, что на участке $0 - 12\text{ c}$ вертикальная составляющая скорости равна 
\begin{equation}
\frac{\Delta p}{\rho g \Delta t} = \frac{240\text{ кПа}}{10\text{ м/с}^2 \cdot 1000\text{ кг/м}^3 \cdot 12\text{ с}} = 2\text{ м/с}.
\end{equation} 
Она равна полной скорости дельфина, а это значит, что он дигался вертикально вниз. Нетрудно заметить, что такой же коэффициент наклона с точностью до знака имеют все наклонные участки. Значит вклад в перемещение дельфина вносят только горизонтальные участки графика, на которых он двигался строго горизонтально. Тогда нетрудно найти его перемещение
\begin{equation}
	\Delta x  = 2\text{ м/с} \cdot (20\text{ с} + 20\text{ с} + 28\text{ с}) = 136\text{ м}.
\end{equation} 
\olympanswer{136\text{ м}.}

\ifgrade
\begin{grade-env}
	\grade{1}{Формула связи изменения давления со скоростью}
	\grade{1}{Утверждение о том, что на наклонных участках движение строго вертикально,}
	\grade{2}{Ответ.}
\end{grade-env}
\fi
