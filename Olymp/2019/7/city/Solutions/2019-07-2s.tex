Для начала поймем, что оптимальным (с точки зрения затраченного времени) будет движение, при котором улитки оказываются на вершине одновременно. В противном случае первая из добравшихся улиток могла потратить больше времени, неся свой домик, тем самым сократив время пути второй улитки, и общее время было бы меньше.

Во вторых, заметим, что быстрой улитке невыгодно двигаться вверх без домика, так как потом ей придется дольше ждать медленную, которой придется этот участок пути пройти, неся домик.

Значит оптимальный путь для медленой улитки состоит из кусков, на которых она несет домик вверх сама и кусков, на которых она движется вверх без домика. Быстрая улитка может нести домик вверх, либо двигаться вниз без домика, чтобы помочь медленной.

Обозначим скорость медленной улитки $v_1$, а быстрой --- $v_2$.

Пусть медленная улитка прошла путь $S$ с домиком и $L-S$ без домика, где $L$ --- весь путь по вертикали, который улиткам нужно преодолеть. Тогда время, которое для этого потребовалось
\begin{equation}
	t_1 = \frac{S}{v_1} + \frac{L-S}{2v_1}.
\end{equation}

Движение быстрой улитки заключается в том, что ей нужно пронести один домик, спуститься за вторым, когда медленная улитка его оставит и пронести его, как минимум до высоты первого. Тогда суммарное время ее движения будет иметь вид
\begin{equation}
	t_2 = \frac{L}{v_2} + \frac{L-S}{2v_2} + \frac{L-S}{v_2}.
\end{equation}

Приравнивая времена $t_1 = t_2$ получаем уравнение, из которого можно найти $S$. После подстановки скоростей и домножения на $2v_2$ оно принимает вид
\begin{equation}
	8S + 4L -4S = 2L + L - S +2L - 2S.
\end{equation}
Его решение $S = \frac{L}{7}$.

Можем подставить найденое значение в $t_1$ и получить общее время пути, которое по построению является минимальным
\begin{equation}
	t = \frac{L}{7v_1} + \frac{3L}{7v_1} = \frac{4L}{7v_1} \approx 216\text{ ч}.
\end{equation}

\olympanswer{$216$ ч.}

\ifgrade
\begin{grade-env}
	\grade{1}{Идея о равенстве времен,}
	\grade{1}{Утверждение о том, из чего состоит движение быстрой улитки,}
	\grade{1}{Выражение для нахождения $S$,}
	\grade{1}{Ответ.}
\end{grade-env}
\fi
