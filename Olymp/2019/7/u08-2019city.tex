\documentclass[11pt, a4paper, onecolumn]{article}
\usepackage{amssymb,amsmath,amsfonts}
\usepackage{bm}
\usepackage[margin = 1.5cm,
		    marginparsep = 0.2cm,
			headsep = 0.5cm,
			footskip = 0.7cm, 	
			marginparwidth = 1cm]{geometry}
\pagestyle{empty}
\setlength{\parindent}{0pt}
\setlength{\parskip}{0pt}

\usepackage{indentfirst,misccorr}
\usepackage{calc}

% шрифты для разных комипляторов
\usepackage{ifpdf}
\usepackage{ifluatex}
\ifpdf
	\ifluatex % if LuaLaTex, then
		\usepackage{unicode-math}
		\usepackage{luatextra} % LuaLaTeX fontspec и макросы
		\usepackage{polyglossia}  
		\usepackage{indentfirst}
		\setdefaultlanguage{russian} 
		\setotherlanguage{english}
		\defaultfontfeatures{Renderer=Basic, Ligatures=TeX} % для работы лигатур
		% настройка шрифтов для LuaLaTeX
		\newfontfamily\cyrillicfont{Linux Libertine}
		\setmainfont{Linux Libertine}
		\setmonofont{Consolas}
		\setsansfont{Linux Biolinum O}
		\setmathfont{Latin Modern Math}
		\usepackage[unicode]{hyperref}
	\else % if pdflatex, then
		\usepackage[utf8]{inputenc}
		\usepackage[T2A]{fontenc}
		\usepackage[russian]{babel}
		\usepackage[unicode]{hyperref}
		\usepackage[pdftex]{graphicx}
%		\usepackage{cmlgc}
	\fi
\else % if xelatex, then
	\usepackage{unicode-math}
	% xelatex specific packages
	\usepackage[xetex]{hyperref}
	\usepackage{xltxtra} % \XeLaTeX macro
	\usepackage{xunicode} % some extra unicode support
	\defaultfontfeatures{Mapping = tex-text}
	\usepackage{polyglossia} % instead of babel in xelatex
	\usepackage{indentfirst} %
	\setdefaultlanguage[spelling = modern]{russian}
	\setotherlanguage{english} %% объявляет второй язык документа
	% настройка шрифтов для XeLaTeX происходит тут
	\newfontfamily\cyrillicfont{Linux Libertine}
	\setmainfont{Linux Libertine}
	\setmonofont{Consolas}
	\setsansfont{Linux Biolinum O}
	\setmathfont{Latin Modern Math}
\fi


% УСЛОВИЯ ЗАДАЧ
\usepackage{tabularx}
\usepackage{makecell}
\usepackage{enumitem}
% счётчик задач
\newcounter{notask}
\setcounter{notask}{1}

% \task{УСЛОВИЕ ЗАДАЧИ}
% задача без картинки
% оформлена как таблица с двумя колонками
% ширина первой колонки (номер столбца) фиксирована, 0.3cm
% ширина второй колонки автоматически рассчитывается из ширины
% страницы (с учётом всевозможных отступов)
\newcommand{\task}[1]{
	\begin{tabularx}{\textwidth}{|c|X|}
		\cline{1-2}
		\makecell*[{{p{0.5cm}}}]{ \centering \arabic{notask}} &
		\makecell*[{{p{\hsize}}}]{ #1 } \\
		\cline{1-2}
	\end{tabularx}
	
	\vspace{-1pt}
	
	\addtocounter{notask}{1}
}


% \taskpic[ШИРИНА КАРТИНКИ]{УСЛОВИЕ ЗАДАЧИ}{КАРТИНКА}
% задача с картинкой
% оформлена как таблица с тремя колонками
% первый аргумент - необязательный, по умолчанию ширина картинки равна
% 4cm, но можно выставить свою
% ширина второй колонки (условие задачи) рассчитывается из ширины
% страницы и ширины картинки
\newcommand{\taskpic}[3][4cm]{
	\begin{tabularx}{\textwidth}{|c|X|c|}
		\cline{1-3}
		\makecell*[{{p{0.5cm}}}]{ \centering \arabic{notask}} &
		\makecell*[{{p{\hsize}}}]{ #2 } &
		\makecell*[{{p{#1}}}]{ \centering #3} \\
		\cline{1-3}
	\end{tabularx}
	
	\vspace{-1pt}
	
	\addtocounter{notask}{1}
}
 %% Главный файл с разметкой страницы шрифтами и пакетами
\input{../../input/preamble_tikz.tex} %% Картиночки
%% Флаги, чтобы писать два варианта в одном файле. 
\newif\iffirst  %% Флаг для первого варианта
\newif\ifsecond %% Флаг для второго варианта
\newcommand{\variants}[2]{\iffirst #1\fi\ifsecond #2\fi}
\usepackage{ifthen}
\textheight = 790pt			 
%% Рисование районного тура с двумя шапками:
%% Следует использовать так:
%% 		\OlympSetReg{ год в формате 16/17 }{ класс }{ номер варианта }{ условие }
\newcommand{\OlympSetAll}[4][spb]{
	\ifthenelse{\equal{#1}{spb}}
	{\OlympSetRegSPb{#2}{#3}{1}{#4}
	 \OlympSetRegSPb{#2}{#3}{2}{#4}}
	{\OlympSetRegRussia{#2}{#3}{1}{#4}
	 \OlympSetRegRussia{#2}{#3}{2}{#4}}
}
\newcommand{\OlympSetRegRussia}[4]{  %% Районный тур всероссийской
	\ifnum #3=1
		\firsttrue\secondfalse
	\else
		\firstfalse\secondtrue
	\fi
	\setcounter{notask}{1}
	\begin{center}
		\textsc{ Всероссийская олимпиада школьников по физике 20#1 г. } \\
		\textsc{ Районный этап } \\
		\textit{ Решения см. на сайте { \underline{www.physolymp.spb.ru} } } \\
	\end{center}
	\vspace{ -1cm }
	\parbox{ 0.5\textwidth }{ \flushleft  \textsc{#2 класс} }
	\parbox{ 0.5\textwidth }{ \flushright \textsc{#3-й вариант } } \\[ 0.1cm ]
	#4
	\vfill
	\begin{center}
		\textsc{Оставьте условие себе!}
	\end{center}
		\begin{center}
		\scriptsize{\textsc{Отпечатано в РИС ГБНОУ <<СПБ ГДТЮ>> 20\rule{10pt}{0.5pt} г. Заказ \rule{30pt}{0.5pt} тираж \rule{15pt}{0.5pt} экз.}}
	\end{center}
	\clearpage
}

\newcommand{\OlympSetRegSPb}[4]{ 	
	\ifnum #3=1
		\firsttrue\secondfalse
	\else
		\firstfalse\secondtrue
	\fi
	\setcounter{notask}{1}
	\begin{center}
		\textsc{ Городская открытая олимпиада школьников по физике  20#1 г. } \\
		\textsc{ Отборочный этап } \\
		%\textsc{ I городской тур } \\
		\textit{ Решения см. на сайте { \underline{www.physolymp.spb.ru} } } \\
	\end{center}
	\vspace{ -1cm }
	\parbox{ 0.5\textwidth }{ \flushleft  \textsc{#2 класс } }
	\parbox{ 0.5\textwidth }{ \flushright \textsc{#3-й вариант } }\\[ 0.1cm ]
	#4
	\vfill
	\begin{center}
		\textsc{Оставьте условие себе!}
	\end{center}
%		\begin{center}
%		\scriptsize{\textsc{Отпечатано в РИС ГБНОУ <<СПБ ГБДТЮ>> 20\rule{10pt}{0.5pt} г. Заказ \rule{30pt}{0.5pt} тираж \rule{15pt}{0.5pt} экз.}}
%	\end{center}
	\clearpage	
}


%% Рисование городского тура с выводом

\newcommand{\OlympHeader}[4]{
	\setcounter{notask}{1}
	\begin{center}
		\textsc{  Городская открытая олимпиада школьников по физике 20#1 г. } \\
		%\textsc{ Основной этап } \\
		\textsc{ Теоретический тур } \\
		\textit{ Решения см. на сайте { \underline{www.physolymp.spb.ru} } } \\
	\end{center}
	\vspace{ -1cm }
	\parbox{ 0.5\textwidth }{ \flushleft  \textsc{#2 класс} }
	\parbox{ 0.5\textwidth }{ \flushright  \textsc{Первый этап} }
		#3
	\vfill
	\begin{center}
		\textsc{Оставьте условие себе!}
	\end{center}
	\clearpage
	
	\begin{center}
		\textsc{  Городская открытая олимпиада школьников по физике 20#1 г. } \\
		%\textsc{ Основной этап } \\
		\textsc{ Теоретический тур } \\
		\textit{ Решения см. на сайте { \underline{www.physolymp.spb.ru} } } \\
	\end{center}
	\vspace{ -1cm }
	\parbox{ 0.5\textwidth }{ \flushleft  \textsc{8 класс} }
	\parbox{ 0.5\textwidth }{ \flushright  \textsc{Второй этап} }
		#4
	\vfill
	\begin{center}
		\textsc{Оставьте условие себе!}
	\end{center}
	\clearpage	
}
 %% Оформление заголовков
\usepackage{xfrac}

\begin{document}
\OlympHeader{18/19}{8}{

	\taskpic{Из двух тонких цилиндров одинаковой формы, с плотностями $\rho_1$ и $\rho_2$, собрали <<поплавок>> (смотри рисунок) и закрепили его в некоторой точке. Нарисуйте график зависимости угла с горизонталью, под которым будет плавать стержень, от положения точки закрепления. \\
	Плотность воды $\rho_0$, силой сопротивления воды пренебречь, считайте, что в любой момент времени поплавок целиком погружён в воду. }{
	\begin{tikzpicture}
		\fill [pattern = water] (-1.2, -.8) rectangle (1.2, 1);
		\draw [->] (.8, .2) arc (0 : 60 : .8);
		\draw [->] (-.8, 0) arc (180 : 240 : .5);
		\draw [thick, pattern = north west lines] (0,0) rectangle node [midway, above, fill = white, rounded corners = 2pt, inner sep = 1pt, outer sep = 4pt] {$\rho_1$} (1, .2);
		\draw [thick, pattern = north east lines] (0,0) rectangle node [midway, above, fill = white, rounded corners = 2pt, inner sep = 1pt, outer sep = 4pt] {$\rho_2$} (-1, .2);
		\filldraw (-.3, .1) circle (.05);
	\end{tikzpicture}
	}
	\task{задача про лягух от Вани}
	\task{задача про шары от Тимофея}
	\task{
	Для исследования свойств морской воды учёные закрепили на спине тюленя всевозможные датчики и отпустили его на свободу в море. На рисунке приведены показания датчика давления в зависимости от времени. Определите перемещение тюленя за время наблюдения. Известно, что он не поворачивал направо или налево. Во всех направлениях его скорость постоянна и равна  $1{,}5$ м/с, плотность морской воды $1000$~кг/м$^3$, ускорение свободного падения~--- $10$~м/c$^2$, атмосферное давление~--- $100$~кПа.\\
	В ходе исследований ни один тюлень не пострадал.
	}
	\begin{center}
\begin{tikzpicture}[scale=0.92,
	/pgfplots/axis labels at tip/.style={ 
		xlabel style={ at={ (current axis.right of origin) }, 
			xshift = 10 ex,
			yshift = 1 ex,
			anchor = north east,
			fill=white}, 
		ylabel style={ at={ (current axis.above origin) }, 
			xshift = 1 ex,
			yshift = 5 ex,
			anchor = north west,
			fill = white} } ]
	\begin{axis}[
	grid style={line width=.1pt, draw=gray!80},
		line width = 2pt, 
		axis x line = middle,
		axis y line = middle,
		axis labels at tip,
		xmin = 0, xmax = 110,
		ymin = 100, ymax = 330,
		xtick = {0, 20, ..., 100},
		ytick = {100, 150, ..., 300},
		xlabel = {$t,\text{ c }$},
		ylabel = {$p,\text{ кПа}$},
		grid = both,
		major grid style={line width = 1.3pt,draw=black!50},
		minor grid style={line width=.4pt, draw=black!50},
		major tick length = 7pt,
	    every major tick/.style={
			black,
	        line width = 2pt,
        },
		minor tick length = 4pt,
	    every minor tick/.style={
			black,
	        line width = 1pt,
        },
		minor tick num = 0]
	\draw [line width = 2pt] 
		(0, 100) to (14, 310) to (30, 310) to (60, 240) to (84, 240) to (100, 100);
	\end{axis}
	\end{tikzpicture}
	\end{center}

}{
	\taskpic{ До какого уровня $h$ надо заполнить форму длинный жёлоб, чтобы горизонтальная составляющая силы давления воды на участок каркаса $AB$, была в два раза больше (меньше?) вертикальной? \\
	Нижняя часть желоба представляет из себя полуокружность радиуса $r=0{,}56$~см, считайте, что $h > r$. }{
	\begin{tikzpicture}
		\fill [pattern = water] (0,1) to ++ (0, -.5) arc ( -180 : 0 : .5) to ++ (0, .5);
		\draw [very thick] (0,1.5) to (0, 1) to ++ (0, -.5) arc ( -180 : 0 : .5) to ++ (0, .5) to ++ (0, .5);
		\draw [water edge] (0, 1) to (1,1);
		\draw [platform] (1.5,0) to (-.5, 0);
		\draw [<->] (1.2, 0) to node [midway, right] {$h$} ++ (0, 1);
		\draw [->] (.5, .5) to node [pos = .7, above,  fill = white, rounded corners = 2pt, inner sep = 1pt, outer sep = 4pt] {$r$} ++ (-40 : .5);
		\filldraw (.5, .5) circle (.04);
		\filldraw (0, 1.5) circle (.05) node [left] {$A$} (.5,0) circle (.05) node [below=.2] {$B$};
	\end{tikzpicture}
	}
	
	\taskpic{Трактор некоторым образом проехал из точки $A$ в точку $B$ так, что его средняя путевая скорость оказалась в $5/3$ раза больше, чем средняя скорость, рассчитанная по перемещению. По прямой дороге, параллельной отрезку $AB$ движется поток машин. При каком наименьшем расстоянии $L$ между отрезком $AB$ и дорогой трактор гарантированно не мог оказаться на дороге, в процессе своего движения? Расстояние от $A$ до $B$ равно $1$~км.}{
	\begin{tikzpicture}
		\filldraw (0,0) circle (.05) node [left] {$A$};
		\filldraw (0,2) circle (.05) node [left] {$B$};
		\draw [dashed] (0, 0) to (0, 2);
		\draw [] (1, 0) to ++ (0, 2) (1.5, 0) to ++ (0, 2); 
		\draw (1.3, .95) node [rotate = 90] {дорога};
		\draw [<->] (0, 1) to node [midway, above] {$L$} ++ (1, 0);
	\end{tikzpicture}
	}
	
	\taskpic{<<Коромысло>> представляет из себя стержень, к одному концу которого подвешено девятилитровое ведро. Известно, что если <<коромысло>> лежит на столе так, что точка крепления ведра находится на расстоянии $x=30$~см от края стола, то в ведро можно налить не более $3$~кг воды~--- иначе нарушится равновесие. Если же стержень выдвинут на $x=20$~см, можно налить не более $6$~кг  воды. Наконец, если стержень выдвинут на $x=10$~см, то в ведро удастся налить $9$~кг воды. Какова максимальная масса воды в ведре, которую можно уравновесить на краю стола при помощи <<коромысла>>, если стержень выдвинут? 
	{\LARGE Дать точку на гиперболе, L, полное ведро с запасом, спросить про точку на плато} }{
	\begin{tikzpicture}
		\pgfmathsetmacro{\r}{-.5*cos(120)}
		\draw [platform] (1,-1) to (1,0) to (-.7,0);
		\draw [thick, fill = black] (-.5,0) rectangle (1+1, .08);
		\draw [thick] (2, .05) to ++ (0, -.4)  coordinate (A) to ++ (-120 : .5) coordinate (L) ++ (.5, 0) coordinate (R) to (A);
		\draw [fill = black!5] (R) arc (0 : -180 : .25 and .085) to ++ (.1, -.4) arc (-180 : 0 : .15 and .051) to (R) to (A);
		\draw (R) arc (0 : 180 : .25 and .085);
		\draw [<->] (1,.2) to node [midway, above] {$x$} ++ (1,0);
	\end{tikzpicture}
	}
}
\end{document}