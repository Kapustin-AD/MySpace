\usepackage{tikz}
\usepackage{pgfplots}
\usepackage{pgfmath}
\usepackage{wrapfig}
\usepackage{subcaption}
\usepackage{tkz-euclide}
\usetkzobj{all}

% основные библиотеки tikz
\usetikzlibrary{
    arrows,
    arrows.meta,
    bending,
	calc,
	patterns,
	intersections,
	decorations.pathreplacing,
	decorations.pathmorphing,
	decorations.text,
	decorations.markings,
	shapes,
	positioning,
    through,
    quotes,
    angles,
    dateplot
}

% библиотеки для электричества
\usetikzlibrary{circuits.ee, circuits.ee.IEC}

% для работы с графиками            
\pgfplotsset{compat = newest}

% основные используемые стили
% стиль для стрелки
\tikzset{
    >=latex,
    % платформа: пол или потолок  
    interface/.style = {
        postaction = {
            draw,
            decorate,
            decoration = {
                border,
                angle = 45,
                amplitude = 0.2cm,
                segment length = 1mm
            }
        }
    },
    perpinterface/.style = {
        postaction = {
            draw,
            decorate,
            decoration = {
                ticks,
                raise = 0.07cm,
                amplitude = 0.07cm,
                segment length = 1mm
            }
        }
    },
    % пружина
    spring/.style = {
        decorate,
        decoration = {
            coil,
            amplitude = 1mm,
            segment length = 1mm
        }
    },
	% заряд, вершина, просто точка
    dot/.style = {
        inner sep = 0mm,
        minimum size = 0.18cm,
        fill,
        circle
    },
    % граница воды
    water edge/.style = {
    	draw,
		decorate,
		decoration = {
			snake,
			amplitude = 0.4,
			segment length = 10
		}
    },
	% стрелка в середине отрезка
    marrow/.style = {
        postaction = {
            draw,
            decorate,
            decoration = {
                markings,
                mark = at position 0.6 with {\arrow{latex}}
            }
        }
    }
}


\tikzstyle{dotnode} = [draw, fill, inner sep = 0pt, minimum size = 1mm, circle]
\tikzstyle{termnode} = [draw, fill = white, inner sep = 0pt, minimum size = 1.5mm, circle]

\newcommand{\termcircuit}[2]{
	\node[termnode] at ({#1}, {#2}) {};
    \draw ({#1 + 0.12}, {#2 + 0.15}) -- ({#1 - 0.12}, {#2 - 0.15});
}




% амперметр
\tikzset{circuit declare symbol = ammeter}
\tikzset{
    set ammeter graphic = {
        draw,
        generic circle IEC,
        minimum size = 5mm,
        fill = white,
        info = center:A
    }
} 

% вольтметр
\tikzset{circuit declare symbol = voltmeter}
\tikzset{
    set voltmeter graphic = {
        draw,
        generic circle IEC,
        minimum size = 5mm,
        fill = white,
        info = center:V
    }
} 

% кружок
\tikzset{circuit declare symbol = meter}
\tikzset{
    set meter graphic = {
        draw,
        generic circle IEC,
        minimum size = 5mm,
        fill = white
    }
} 


\newcommand{\platform}[3][fill = none]{
	\draw [draw = none, #1 ] 
		#2 -- ($#2!0.15cm!90:#3$) -- ($#3!0.15cm!-90:#2$) -- #3;
	\fill [pattern = north west lines] 
		#2 -- ($#2!0.15cm!90:#3$) -- ($#3!0.15cm!-90:#2$) -- #3;
	\draw [very thick] #2 -- #3
}

% pattern для отрисовки воды
\pgfdeclarepatternformonly{water}
    {\pgfqpoint{0pt}{0pt}}
    {\pgfqpoint{6pt}{6pt}}
    {\pgfqpoint{6pt}{6pt}}
    {
        \pgfsetlinewidth{0.3pt}
        \pgfpathmoveto{\pgfqpoint{0pt}{0pt}}
        \pgfpathlineto{\pgfqpoint{2pt}{0pt}}
        \pgfpathmoveto{\pgfqpoint{3pt}{3pt}}
        \pgfpathlineto{\pgfqpoint{6pt}{3pt}}
        \pgfusepath{stroke}
    }

% формат чисел в графиках \begin{axis}...\end{axis}
\tikzset{/pgf/number format/use comma}
