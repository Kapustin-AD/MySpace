Вдоль вертикального стержня без трения могут двигаться $3$ бусины. Сначала каждую держат на определенной высоте (см. рис.) а затем одновременно отпускают. Определите сколько раз центральная бусина столкнется с остальными бусинами, до того как она...

Бусины одинаковые и сталкиваются абсолютно упруго, то есть при столкновении они ''обмениваются`` скоростями. При этом, если до столкновения бусина $1$ имела скорость $\vec{v_1}$, а бусина $2$ имела $\vec{v_2}$, то сразу после столкновения бусина $1$ будет иметь скорость $\vec{v_2}$, а бусина $2$ будет иметь $\vec{v_1}$. Ускорение свободного падения $g$. 