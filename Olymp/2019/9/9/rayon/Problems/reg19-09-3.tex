В подвале общежития установлен отопительный котёл, снабжающий горячей водой душевую. В котёл поступает вода при температуре $10^\circ\mathrm{C}$. Нагреватель постоянной мощности работает только когда температура воды в котле меньше определённого значения. Когда душ принимают $3$ человека, температура в котле равна $80^\circ\mathrm{C}$, а когда $7$ человек --- $60^\circ\mathrm{C}$. Чему равна температура в котле, когда душ принимают $6$ человек? Считайте, что расход горячей воды одним человеком не зависит от её температуры, и вода в котле быстро перемешивается.
