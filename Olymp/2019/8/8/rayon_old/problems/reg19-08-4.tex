Неопытный водитель перемещается из точки $A$ в точку $B$ таким образом, что его средняя путевая скорость в $5/3$ раза больше, чем средняя скорость, рассчитанная по перемещению. По прямой, параллельной отрезку $AB$ движется поток машин. При каком наименьшем расстоянии между отрезком $AB$ и прямой столкновение гарантированно не произойдёт, если расстояние от $A$ до $B$ равно $L$?