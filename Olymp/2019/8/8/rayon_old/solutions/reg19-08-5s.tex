\par В условии сказано, что $22$~августа упало $10$ яблок, из графика видно, что средняя масса яблок в этот день была равна $135$~г, поэтому в тот день упало $135~\text{г}\cdot10=1{,}35~\text{кг}$. Проделаем такую же процедуру для каждого дня и перестроим график так, чтобы получить зависимость массы выпавших яблок от дня.
\begin{center}	 
\begin{tikzpicture}[
	 	/pgfplots/axis labels at tip/.style={ 
	 		xlabel style={ at={ (current axis.right of origin) }, 
	 			yshift = 1 ex,
	 			anchor = south east,
	 			fill=white}, 
	 		ylabel style={ at={ (current axis.above origin) }, 
	 			xshift = 1 ex, 
	 			anchor = north west,
	 			fill = white} } ]
	  \begin{axis}[
	  	date coordinates in=x,
	  	date ZERO=2018-08-21,
	    xticklabel=\day.\month,
	    ymin=0.1, ymax=3.92,
		xmin=2018-08-21,
		xmax=2018-09-21,
		xticklabel style={
		rotate=90,
			anchor=near xticklabel,
		},
		line width = 2pt, 
		axis x line = middle,
		axis y line = middle,
		axis labels at tip,		       	  
	    xlabel = {день},
		ylabel = {$m$, кг},
		grid = both,
		major grid style={line width = 1.3pt,draw=black!50},
		minor grid style={line width=.4pt, draw=black!50},
		major tick length = 7pt,
	    every major tick/.style={
			black,
	        line width = 2pt,
        },
		minor tick length = 4pt,
	    every minor tick/.style={
			black,
	        line width = 1pt,
        },
		minor tick num = 4
	   ]
	   \addplot [only marks] coordinates  {
	    (2018-08-22, 0.675)
	    (2018-08-23, 0.819)
	    (2018-08-24, 0.973)
	    (2018-08-25, 1.136)
	    (2018-08-26, 1.296)
	    (2018-08-27, 1.444)
	    (2018-08-28, 1.558)
	    (2018-08-29, 1.662)
	    (2018-08-30, 1.764)
	    (2018-08-31, 1.867)
	    (2018-09-01, 1.969)
	    (2018-09-02, 2.070)
	    (2018-09-03, 2.172)
	    (2018-09-04, 2.274)
	    (2018-09-05, 2.375)
	    (2018-09-06, 2.476)
	    (2018-09-07, 2.577)
	    (2018-09-08, 2.678)	    
	    (2018-09-09, 2.779)
	    (2018-09-10, 2.880)
	    (2018-09-11, 2.980)
	    (2018-09-12, 3.081)	
	    (2018-09-13, 3.182)
	    (2018-09-14, 3.282)
	    (2018-09-15, 3.383)
	    (2018-09-16, 3.465)	        
	   };
 \end{axis}
\end{tikzpicture}
\end{center}
\par Теперь чтобы узнать суммарную массу яблок достаточно посчитать площадь под графиком в период с $27$~августа по $15$~сентября. Она равна $48$~кг яблок.
