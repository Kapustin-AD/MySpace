\par В условии сказано, что количество тепла в единицу времени, которое уходит в окружающую среду через единицу площади, пропорционально разности температур. Рассмотрим некоторый маленький промежуток времени $\Delta t$, в течении которого температуру тела можно считать постоянной. Тепло, которое тело отдало в окружающую среду равно:
\begin{equation}
	\Delta Q = \alpha \cdot S \cdot (T_\text{тело} - T_\text{комнаты}) \cdot \Delta t ,
\end{equation}
где $\alpha$~--- некоторый постоянный коэффициент, $S$~--- площадь, через которую уходит тепло.
\par  Пускай удельная теплоёмкость тела равна $c$, тогда этот же закон переписывается в виде:
\begin{equation}
	\frac{\Delta T}{\Delta t} = \frac{\alpha}{c}  \cdot \frac{S}{m} \cdot (T_\text{тело} - T_\text{комнаты}).
\end{equation}
Отсюда следует, что скорость изменения температуры прямо пропорционально зависит от отношения $S/m$. Посчитаем эту величину для кубика, учитывая, что тепло уходит только через поверхность, которая граничит с воздухом.
\begin{equation}
	\frac{S_\text{кубик}}{m_\text{кубик}} = \frac{5 a^2}{a^3\rho} = 5\cdot \frac{1}{a\rho}.
\end{equation}
Проделав то же для пирамиды, получим:
\begin{equation}
	\frac{S_\text{кубик}}{m_\text{кубик}}
	=
	\frac{\left[5 (3a)^{2} - (2a)^2 \right] 
		 + \left[5 (2a)^2 - a^2\right]
		  + 5a^2 }
		 {(3a)^3\rho + (2a)^3\rho + a^3 \rho} = \frac{65}{32} \cdot \frac{1}{a\rho}.
\end{equation}
\par Конструкции изготовлены из одного материала, поэтому равны их плотности и удельные теплоёмкости, значит можно посчитать во сколько раз скорость остывания кубика больше,
\begin{equation}
	\frac{ 5 }{\frac{65}{32}} = \frac{32}{13} \approx 2{,}5.
\end{equation}
Поэтому если пирамидка кубик остывает за $10$~секунд, пирамидка остынет за $25$~секунд.
\olympanswer{пирамидка остынет за $25$~секунд}


