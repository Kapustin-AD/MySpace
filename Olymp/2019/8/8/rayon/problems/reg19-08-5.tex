Когда в саду начинали падать яблоки, Вася стал ежедневно записывать в дневник количество упавших яблок и их суммарную массу. По результатам месячных измерений мальчик построил график значений средней массы яблока в день в зависимости от даты. По этому графику определите, сколько килограмм яблок выпало в период \variants{с $27$ августа по $15$ сентября}{с $23$ августа по $11$ сентября}, если $22$ августа выпало \variants{$10$}{$60$} яблок, а в каждый следующий день падало на $2$ яблока \variants{больше}{меньше}, чем в предыдущий.