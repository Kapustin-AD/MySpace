Машина едет по дороге из пункта $A$ в пункт $D$, проезжая пункты $B$ и $C$. Расстояние между соседними пунктами \variants{$60\text{~км}$}{$100\text{~км}$}. Максимально разрешенная скорость на отрезке $AB$ --- \variants{$60\text{~км/ч}$}{$50\text{~км/ч}$}, на $BC$ --- \variants{$90\text{~км/ч}$}{$75\text{~км/ч}$}, на $CD$ --- \variants{$120\text{~км/ч}$}{$100\text{~км/ч}$}. Вдоль дороги стоят камеры, фиксирующие, в какое время мимо них проезжает машина. Первая камера в пункте $A$, вторая посередине между $B$ и $C$, и третья в пункте $D$. Машина проезжает первую в \variants{$13$:$20$}{$16$:$30$}, вторую в  \variants{$14$:$30$}{$19$:$30$} и третью в \variants{$15$:$30$}{$21$:$00$}. На участках между какими камерами машина точно нарушала скоростной режим?