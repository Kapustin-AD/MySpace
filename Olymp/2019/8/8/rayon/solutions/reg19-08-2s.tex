\par Введём координатную ось. Усадьбе сопоставим точку $0$, а дворцу --- $1$. Время будем измерять в днях. Скорость боярина (измеряемая в $\frac{\text{расстояние до дворца}}{\text{1 день}}$) тогда будет равна $v = \variants{\frac{1}{4}}{ \frac{1}{5}}$, а скорость гонца --- $u = 1$.


Первый день уйдёт на доставку приказа немедленно явиться. За второй день боярин проедет $x_0 = \frac{1}{4}~\left[ \frac{1}{5} \right]$ расстояния до дворца. В третий день начинает движение второй гонец. Он встретится с боярином через
\[
	t_1 = \frac{x_{\text{г2}} - x_0}{u + v} = \frac{1 - \frac{1}{4}}{1 + \frac{1}{4}} = \frac{3}{5}~~\left[ \frac{2}{3} \right]
\]
Координата боярина в момент разворота:
\[
	x_{\text{р1}} = x_0 + t_1 v = \frac{1}{4} + \frac{3}{5} \cdot \frac{1}{4} = \frac{2}{5}~~\left[ \frac{1}{3} \right]
\]
Остаток третьего дня ($1-t_1$) боярин едет назад. Его координата к концу третьего дня:
\[
	x_{3} = x_{\text{р1}} - (1-t_1)v = \frac{2}{5} - \left(1 - \frac{3}{5}\right)\cdot \frac{1}{4} = \frac{3}{10}~~ \left[ \frac{4}{15} \right]
\]
В четвёртый день начинает движение третий гонец. Он догонит боярина через
\[
	t_2 = \frac{x_{\text{г3}} - x_{3}}{u - v} = \frac{1 - \frac{3}{10}}{1 - \frac{1}{4}} = \frac{14}{15}~~\left[ \frac{11}{12} \right]
\]
Координата боярина в момент второго разворота:
\[
	x_{\text{р2}} = x_3 - t_2 v = \frac{3}{10} - \frac{14}{15} \cdot \frac{1}{4} = \frac{1}{15}~~\left[ \frac{1}{12} \right]
\]
Остаток четвёртого дня ($1-t_2$) боярин едет вперёд. Его координата к концу четвёртого дня:
\[
	x_{4} = x_{\text{р2}} + (1-t_2)v = \frac{1}{15} + \left(1 - \frac{14}{15}\right)\cdot \frac{1}{4} = \frac{1}{12}~~ \left[ \frac{1}{10} \right]
\]
Чтобы доехать до дворца, боярину потребуется ещё
\[
	t_3 = \frac{1-x_4}{v} = \frac{1-\frac{1}{12}}{\frac{1}{4}} = \frac{11}{3}~~\left[ \frac{9}{2} \right]
\]
Всего с момента отправления первого гонца пройдёт $T = 4 + t_3 = 7\frac{2}{3}~~\left[ 8\frac{1}{2} \right]$ дня.
\olympanswer{ Боярин приедет во дворец через \variants{$7{,}6$~дней.}{$8{,}5$~дней.} }

