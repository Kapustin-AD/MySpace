	За время движения из точки $A$ в точку $B$ величина перемещения водителя равна длине отрезка $AB$. Обозначим ее за $d$. При этом известно, что путь должен быть больше в определенное число раз, то есть $s = \frac{5}{3}d$. Заметим, что тогда водитель не мог оказаться в таких точках, сумма расстояний от которых до $A$ и $B$ больше чем $s$. 
	\[
		s_A + s_B \geqslant s.
	\]
	И наоборот, всилу неопределенности его траектории, она могла проходить через любую точку, не принадлежащую вышеуказанной области. 
	\begin{figure}[h!]
		\centering
		\begin{tikzpicture}
			\fill[pattern = north east lines] (-3, 3) rectangle (3, -3);
			\fill[white] (0, 0) ellipse (1 and 2);
			\fill (0, -1)node [left] {$A$} circle (0.07);
			\fill (0, 1)node [left] {$B$} circle (0.07);
			\draw[thick] (0, 0) ellipse (1 and 2);
			\draw (0, -1) -- (0.97, -0.5)node [midway, above ] {$s_A$};
			\draw (0.97, -0.5) -- (0, 1)node [midway, left] {$s_B$};
		\end{tikzpicture}
	\end{figure}
	
	Тогда, чтобы столкновения не произошло, прямая, по которой движется поток машин должна полностью лежать в <<Запрещенной зоне>>, а для того, чтобы расстояние до отрезка было минимальным, она должна касаться границы. Граница задается условием $s_A + s_B = s$, которое определяет эллипс на плоскости, однако знание этого факта не необходимо. Раз прямая должна касаться границы, давайте докажем, что точка касания должна быть равноудалена от $A$ и $B$. Для этого нужно показать, что эта точка будет наиболее удалена от отрезка $AB$.
	\begin{figure}[h!]
		\centering
		\begin{tikzpicture}
			\fill[pattern = north east lines] (-3, 3) rectangle (3, -3);
			\fill[white] (0, 0) ellipse (1 and 2);
			\draw[very thick, red] (1, 3) -- (1, -3); 
			\fill (0, -1)node [left] {$A$} circle (0.07);
			\fill (0, 1)node [left] {$B$} circle (0.07);
			\draw[thick] (0, 0) ellipse (1 and 2);
			\draw (0, -1) -- (1, 0)node[right]{$M$};
			\fill (1, 0) circle (0.07);
			\fill (1, -0.3) circle (0.07)node[right]{$M'$};
			\draw (1, 0) -- (0, 1);
			\draw[dashed] (0, 1) -- (0, -1);
			\draw[dashed] (0, 0) -- (1, 0);
		\end{tikzpicture}
	\end{figure}
	
	Сдвинем точку $M$ на $\Delta x$ вниз. При этом сумма расстояний до точек $A$ и $B$
	\[
		s'_A + s'_B = \sqrt{\left(\frac{d}{2}+\Delta x \right)^2 + y^2} + \sqrt{\left(\frac{d}{2}-\Delta x \right)^2 + y^2} 
	\]
	увеличится, в чем можно убедиться возведениями неравенства в квадрат. Значит для того, чтобы точка $M'$ лежала на границе нужно уменьшить $y$, а следовательно первоначальный $y$ является наибольшим. Найдем его по теореме Пифагора. \\
	\olympanswer{ $y = \frac{4}{3}L$. }