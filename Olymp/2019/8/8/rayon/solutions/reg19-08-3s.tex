Обозначим плечи качелей как $l_1, l_2$, а расстояние между правым концом левых качелей и левым концом правых качелей (то есть длину их перекрытия) за $x$. По условию, $2l_1 = 8$, то есть $l_1 = 4$.

В первой ситуации из условия выигрыш в силе $\frac{l_1}{l_1}\cdot\frac{l_2 - x}{l_2} = \frac12$, а во второй $\frac{l_2}{l_2}\cdot\frac{l_1 - x}{l_1} = \frac14$. Тогда:

\begin{equation}
\left\{\begin{aligned}
1 - \frac x{l_2} = \frac12\\
1 - \frac x{l_1} = \frac14
\end{aligned}\right.
\end{equation}

\begin{equation}
\left\{\begin{aligned}
l_2 = 2x\\
x = \frac34l_1
\end{aligned}\right.
\end{equation}

Значит, $x = 3$ м, $l_2 = 6$ м. Расстояние между опорами тогда $l_1 + l_2 - x = 7$ м.