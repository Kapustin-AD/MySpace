\documentclass[11pt, a4paper, onecolumn]{article}
\usepackage{amssymb,amsmath,amsfonts}
\usepackage{bm}
\usepackage[margin = 1.5cm,
		    marginparsep = 0.2cm,
			headsep = 0.5cm,
			footskip = 0.7cm, 	
			marginparwidth = 1cm]{geometry}
\pagestyle{empty}
\setlength{\parindent}{0pt}
\setlength{\parskip}{0pt}

\usepackage{indentfirst,misccorr}
\usepackage{calc}

% шрифты для разных комипляторов
\usepackage{ifpdf}
\usepackage{ifluatex}
\ifpdf
	\ifluatex % if LuaLaTex, then
		\usepackage{unicode-math}
		\usepackage{luatextra} % LuaLaTeX fontspec и макросы
		\usepackage{polyglossia}  
		\usepackage{indentfirst}
		\setdefaultlanguage{russian} 
		\setotherlanguage{english}
		\defaultfontfeatures{Renderer=Basic, Ligatures=TeX} % для работы лигатур
		% настройка шрифтов для LuaLaTeX
		\newfontfamily\cyrillicfont{Linux Libertine}
		\setmainfont{Linux Libertine}
		\setmonofont{Consolas}
		\setsansfont{Linux Biolinum O}
		\setmathfont{Latin Modern Math}
		\usepackage[unicode]{hyperref}
	\else % if pdflatex, then
		\usepackage[utf8]{inputenc}
		\usepackage[T2A]{fontenc}
		\usepackage[russian]{babel}
		\usepackage[unicode]{hyperref}
		\usepackage[pdftex]{graphicx}
%		\usepackage{cmlgc}
	\fi
\else % if xelatex, then
	\usepackage{unicode-math}
	% xelatex specific packages
	\usepackage[xetex]{hyperref}
	\usepackage{xltxtra} % \XeLaTeX macro
	\usepackage{xunicode} % some extra unicode support
	\defaultfontfeatures{Mapping = tex-text}
	\usepackage{polyglossia} % instead of babel in xelatex
	\usepackage{indentfirst} %
	\setdefaultlanguage[spelling = modern]{russian}
	\setotherlanguage{english} %% объявляет второй язык документа
	% настройка шрифтов для XeLaTeX происходит тут
	\newfontfamily\cyrillicfont{Linux Libertine}
	\setmainfont{Linux Libertine}
	\setmonofont{Consolas}
	\setsansfont{Linux Biolinum O}
	\setmathfont{Latin Modern Math}
\fi


% УСЛОВИЯ ЗАДАЧ
\usepackage{tabularx}
\usepackage{makecell}
\usepackage{enumitem}
% счётчик задач
\newcounter{notask}
\setcounter{notask}{1}

% \task{УСЛОВИЕ ЗАДАЧИ}
% задача без картинки
% оформлена как таблица с двумя колонками
% ширина первой колонки (номер столбца) фиксирована, 0.3cm
% ширина второй колонки автоматически рассчитывается из ширины
% страницы (с учётом всевозможных отступов)
\newcommand{\task}[1]{
	\begin{tabularx}{\textwidth}{|c|X|}
		\cline{1-2}
		\makecell*[{{p{0.5cm}}}]{ \centering \arabic{notask}} &
		\makecell*[{{p{\hsize}}}]{ #1 } \\
		\cline{1-2}
	\end{tabularx}
	
	\vspace{-1pt}
	
	\addtocounter{notask}{1}
}


% \taskpic[ШИРИНА КАРТИНКИ]{УСЛОВИЕ ЗАДАЧИ}{КАРТИНКА}
% задача с картинкой
% оформлена как таблица с тремя колонками
% первый аргумент - необязательный, по умолчанию ширина картинки равна
% 4cm, но можно выставить свою
% ширина второй колонки (условие задачи) рассчитывается из ширины
% страницы и ширины картинки
\newcommand{\taskpic}[3][4cm]{
	\begin{tabularx}{\textwidth}{|c|X|c|}
		\cline{1-3}
		\makecell*[{{p{0.5cm}}}]{ \centering \arabic{notask}} &
		\makecell*[{{p{\hsize}}}]{ #2 } &
		\makecell*[{{p{#1}}}]{ \centering #3} \\
		\cline{1-3}
	\end{tabularx}
	
	\vspace{-1pt}
	
	\addtocounter{notask}{1}
}
