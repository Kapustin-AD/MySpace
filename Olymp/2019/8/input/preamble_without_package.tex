\documentclass[11pt,a4paper,onecolumn]{article}

\usepackage{amssymb,amsmath,amsfonts}
\usepackage{bm}
\usepackage[margin = 1.5cm,
			marginparsep = 0.2cm,
			headsep = 0.5cm,
			footskip = 0.7cm, 	
			marginparwidth = 1cm]{geometry}
%\usepackage{showframe}			
			
\usepackage{indentfirst,misccorr}
\usepackage{calc}
\pagestyle{empty}


\setlength{\parindent}{0pt}
\setlength{\parskip}{0pt}
% ШРИФТЫ
\usepackage{ifpdf, ifxetex,ifluatex}
\ifpdf
	\ifluatex % if LuaLaTex, then
		\usepackage{unicode-math}
		\usepackage{luatextra} % LuaLaTeX fontspec и макросы
		\usepackage{polyglossia}  
		\setdefaultlanguage{russian} 
		\setotherlanguage{english}
		\defaultfontfeatures{Renderer=Basic, Ligatures=TeX} % для работы лигатур
		% настройка шрифтов для LuaLaTeX
		\newfontfamily\cyrillicfont{Linux Libertine}
		\setmainfont{Linux Libertine}
		\setmonofont{Consolas}
		\setsansfont{Linux Biolinum O}
		\setmathfont{Cambria Math}
	\else % if pdflatex, then
		\usepackage[utf8]{inputenc}
		\usepackage[T2A]{fontenc}
		\usepackage[russian]{babel}
		\usepackage[unicode]{hyperref}
		\usepackage[pdftex]{graphicx}
		\usepackage{cmlgc}
	\fi
\else % if xelatex, then
	\usepackage{unicode-math}
	% xelatex specific packages
	\usepackage[xetex]{hyperref}
	\usepackage{xltxtra} % \XeLaTeX macro
	\usepackage{xunicode} % some extra unicode support
	\defaultfontfeatures{Mapping = tex-text}
	\usepackage{polyglossia} % instead of babel in xelatex
	\usepackage{indentfirst} %
	\setdefaultlanguage[spelling = modern]{russian}
	\setotherlanguage{english} %% объявляет второй язык документа
	% настройка шрифтов для XeLaTeX происходит тут
	\newfontfamily\cyrillicfont{Linux Libertine}
	\setmainfont{Linux Libertine}
	\setmonofont{Consolas}
	\setsansfont{Linux Biolinum O}
	\setmathfont{Cambria Math}
\fi




%% Рисование районного тура с двумя шапками:
%% Следует использовать так:
%% 		\OlympSetReg{ год в формате 16/17 }{ класс }{ номер варианта }{ условие }
\newcommand{\OlympSetRegRussia}[4]{  %% Районный тур всероссийской
	\setcounter{notask}{1}
	\begin{center}
		\textsc{ Всероссийская олимпиада школьников по физике 20#1 г. } \\
		\textsc{ Районный этап } \\
		\textit{ Решения см. на сайте { \underline{www.physolymp.spb.ru} } } \\
	\end{center}
	\vspace{ -1cm }
	\parbox{ 0.5\textwidth }{ \flushleft  \textsc{#2 класс} }
	\parbox{ 0.5\textwidth }{ \flushright \textsc{#3-й вариант } } \\[ 0.1cm ]
	#4
	\vfill
	\begin{center}
		\textsc{Оставьте условие себе!}
	\end{center}
	\clearpage
}

\newcommand{\OlympSetRegSPb}[4]{ 	
	\setcounter{notask}{1}
	\begin{center}
		\textsc{ Городская открытая олимпиада школьников по физике  20#1 г. } \\
		\textsc{ Отборочный этап } \\
		%\textsc{ I городской тур } \\
		\textit{ Решения см. на сайте { \underline{www.physolymp.spb.ru} } } \\
	\end{center}
	\vspace{ -1cm }
	\parbox{ 0.5\textwidth }{ \flushleft  \textsc{#2 класс } }
	\parbox{ 0.5\textwidth }{ \flushright \textsc{#3-й вариант } }\\[ 0.1cm ]
	#4
	\vfill
	\begin{center}
		\textsc{Оставьте условие себе!}
	\end{center}
	\clearpage	
}


%% Рисование городского тура с выводом

\newcommand{\OlympHeader}[4]{
	\setcounter{notask}{1}
	\begin{center}
		\textsc{  Городская открытая олимпиада школьников по физике 20#1 г. } \\
		%\textsc{ Основной этап } \\
		\textsc{ Теоретический тур } \\
		\textit{ Решения см. на сайте { \underline{www.physolymp.spb.ru} } } \\
	\end{center}
	\vspace{ -1cm }
	\parbox{ 0.5\textwidth }{ \flushleft  \textsc{#2 класс} }
	\parbox{ 0.5\textwidth }{ \flushright  \textsc{Первый этап} }
		#3
	\vfill
	\begin{center}
		\textsc{Оставьте условие себе!}
	\end{center}
	\clearpage
	
	\begin{center}
		\textsc{  Городская открытая олимпиада школьников по физике 2016/17 г. } \\
		%\textsc{ Основной этап } \\
		\textsc{ Теоретический тур } \\
		\textit{ Решения см. на сайте { \underline{www.physolymp.spb.ru} } } \\
	\end{center}
	\vspace{ -1cm }
	\parbox{ 0.5\textwidth }{ \flushleft  \textsc{8 класс} }
	\parbox{ 0.5\textwidth }{ \flushright  \textsc{Второй этап} }
		#4
	\vfill
	\begin{center}
		\textsc{Оставьте условие себе!}
	\end{center}
	\clearpage	
}



% УСЛОВИЯ ЗАДАЧ
\usepackage{tabularx}
\usepackage{makecell}
\usepackage{enumitem}
% счётчик задач
\newcounter{notask}
\setcounter{notask}{1}

% \task{УСЛОВИЕ ЗАДАЧИ}
% задача без картинки
% оформлена как таблица с двумя колонками
% ширина первой колонки (номер столбца) фиксирована, 0.3cm
% ширина второй колонки автоматически рассчитывается из ширины
% страницы (с учётом всевозможных отступов)
\newcommand{\task}[1]{
	\begin{tabularx}{\textwidth}{|c|X|}
		\cline{1-2}
		\makecell*[{{p{0.5cm}}}]{ \centering \arabic{notask}} &
		\makecell*[{{p{\hsize}}}]{ #1 } \\
		\cline{1-2}
	\end{tabularx}
	
	\vspace{-1pt}
	
	\addtocounter{notask}{1}
}


% \taskpic[ШИРИНА КАРТИНКИ]{УСЛОВИЕ ЗАДАЧИ}{КАРТИНКА}
% задача с картинкой
% оформлена как таблица с тремя колонками
% первый аргумент - необязательный, по умолчанию ширина картинки равна
% 4cm, но можно выставить свою
% ширина второй колонки (условие задачи) рассчитывается из ширины
% страницы и ширины картинки
\newcommand{\taskpic}[3][4cm]{
	\begin{tabularx}{\textwidth}{|c|X|c|}
		\cline{1-3}
		\makecell*[{{p{0.5cm}}}]{ \centering \arabic{notask}} &
		\makecell*[{{p{\hsize}}}]{ #2 } &
		\makecell*[{{p{#1}}}]{ \centering #3} \\
		\cline{1-3}
	\end{tabularx}
	
	\vspace{-1pt}
	
	\addtocounter{notask}{1}
}


% Здесь идут команды для tikz если они будут конфликтными надо расскоментировать следующую строчку
%\endinput
\usepackage{tikz}
\usepackage{pgfplots}
\usepackage{wrapfig}
\usepackage{subfig}

% основные библиотеки tikz
\usetikzlibrary{
	arrows,
	calc,
	patterns,
	intersections,
	decorations.pathreplacing,
	decorations.pathmorphing,
	decorations.text,
	decorations.markings,
	shapes,
	positioning
}

% библиотеки для электричества
\usetikzlibrary{circuits.ee, circuits.ee.IEC}

% для работы с графиками            
\pgfplotsset{compat = newest}

% основные используемые стили
% стиль для стрелки
\tikzset{
	>=latex,
	% платформа: пол или потолок  
	interface/.style = {
		postaction = {
			draw,
			decorate,
			decoration = {
				border,
				angle = 45,
				amplitude = 0.2cm,
				segment length = 1mm
			}
		}
	},
	% пружина  
	spring/.style = {
		decorate,
		decoration = {
			coil,
			amplitude = 1mm,
			segment length = 1mm
		},
		thick
	},
	% заряд, вершина, просто точка
	dot/.style = {
		inner sep = 0mm,
		minimum size = 0.18cm,
		fill,
		circle
	},
	% стрелка в середине отрезка  
	marrow/.style = {
		postaction = {
			draw,
			decorate,
			decoration = {
				markings,
				mark = at position 0.6 with {\arrow{latex}}
			}
		}
	}
}


\tikzstyle{dotnode} = [draw, fill, inner sep = 0pt, minimum size = 1mm, circle]

\tikzstyle{termnode} = [draw, fill = white, inner sep = 0pt, minimum size = 1.5mm, circle]

\newcommand{\termcircuit}[2]{
	\node[termnode] at ({#1}, {#2}) {};
	\draw ({#1 + 0.12}, {#2 + 0.15}) -- ({#1 - 0.12}, {#2 - 0.15});
}




% обозначение угла
\tikzset{
	arcnode/.style = {
		decoration = {
			markings,
			raise = 2mm,
			mark = at position 0.5 with {
				\node[inner sep = 0] {#1};
			}
		},
		postaction = {decorate}
	}
}

% команда для отметки угла: проводит дугу между двумя лучами,
% проведёнными между точками #2--#3 и #2--#4
% первый аргумент - необязательный, стиль линии
% #2,#3,#4 - точки
% #5 - радиус дуги для обозначения угла
% #6 - обозначение угла, например, $\alpha$
\newcommand*\marktheangle[6][]{
	\draw[thick, arcnode = {#6}, #1] let \p2 = ($(#3)-(#2)$),
	\p3 = ($(#4)-(#2)$),
	\n2 = {atan2(\x2,\y2)},
	\n3 = {atan2(\x3,\y3)}%
	in ($(\n2:#5)+(#2)$) arc (\n2:\n3:#5);
}


% амперметр
\tikzset{circuit declare symbol = ammeter}
\tikzset{
	set ammeter graphic = {
		draw,
		generic circle IEC,
		minimum size = 5mm,
		info=center:A
	}
} 

% вольтметр
\tikzset{circuit declare symbol = voltmeter}
\tikzset{
	set voltmeter graphic = {
		draw,
		generic circle IEC,
		minimum size = 5mm,
		info=center:V
	}
} 

% кружок
\tikzset{circuit declare symbol = meter}
\tikzset{
	set meter graphic = {
		draw,
		generic circle IEC,
		minimum size = 5mm
	}
} 


\newcommand{\platform}[2]{
	\draw [draw = none, pattern = north west lines] 
	#1 -- ( $#1!0.15cm!90:#2$ ) -- ( $#2!0.15cm!-90:#1$ ) -- #2;
	\draw #1 -- #2
}

